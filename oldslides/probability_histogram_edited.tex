   \documentclass[handout]{beamer}



   \mode<presentation>
   {
     \usetheme{PaloAlto}
   \setbeamertemplate{footline}[page number]

     \setbeamercolor{sidebar}{bg=white, fg=black}
     \setbeamercolor{frametitle}{bg=white, fg=black}
     % or ...
     \setbeamercolor{logo}{bg=white}
     \setbeamercolor{block body}{parent=normal text,bg=white}
     \setbeamercolor{author in sidebar}{fg=black}
     \setbeamercolor{title in sidebar}{fg=black}


     \setbeamercolor*{block title}{use=structure,fg=structure.fg,bg=structure.fg!20!bg}
     \setbeamercolor*{block title alerted}{use=alerted text,fg=alerted text.fg,bg=alerted text.fg!20!bg}
     \setbeamercolor*{block title example}{use=example text,fg=example text.fg,bg=example text.fg!20!bg}


     \setbeamercolor{block body}{parent=normal text,use=block title,bg=block title.bg!50!bg}
     \setbeamercolor{block body alerted}{parent=normal text,use=block title alerted,bg=block title alerted.bg!50!bg}
     \setbeamercolor{block body example}{parent=normal text,use=block title example,bg=block title example.bg!50!bg}

     % or ...

     \setbeamercovered{transparent}
     % or whatever (possibly just delete it)
     \logo{\resizebox{!}{1.5cm}{\href{\basename{R}}{\includegraphics{image}}}}
   }

   \mode<handout>
   {
     \usetheme{PaloAlto}
     \usecolortheme{default}
     \setbeamercolor{sidebar}{bg=white, fg=black}
     \setbeamercolor{frametitle}{bg=white, fg=black}
     % or ...
     \setbeamercolor{logo}{bg=white}
     \setbeamercolor{block body}{parent=normal text,bg=white}
     \setbeamercolor{author in sidebar}{fg=black}
     \setbeamercolor{title in sidebar}{fg=black}
     \setbeamercovered{transparent}
     % or whatever (possibly just delete it)
     \logo{}
   }

   \usepackage{epsdice}
   \usepackage[latin1]{inputenc}
   \usepackage{graphics}
   \usepackage{amsmath,eepic,epic}

   \newcommand{\figslide}[3]{
   \begin{frame}
   \frametitle{#1}
     \begin{center}
     \resizebox{!}{2.7in}{\includegraphics{#2}}    
     \end{center}
   {#3}
   \end{frame}
   }

   \newcommand{\fighslide}[4]{
   \begin{frame}
   \frametitle{#1}
     \begin{center}
     \resizebox{!}{#4}{\includegraphics{#2}}    
     \end{center}
   {#3}
   \end{frame}
   }

   \newcommand{\figwref}[1]{
   \href{#1}{\tiny \tt #1}}

   \newcommand{\B}[1]{\beta_{#1}}
   \newcommand{\Bh}[1]{\widehat{\beta}_{#1}}
   \newcommand{\V}{\text{Var}}
   \newcommand{\Cov}{\text{Cov}}
   \newcommand{\Vh}{\widehat{\V}}
   \newcommand{\s}{\sigma}
   \newcommand{\sh}{\widehat{\sigma}}

   \newcommand{\argmax}[1]{\mathop{\text{argmax}}_{#1}}
   \newcommand{\argmin}[1]{\mathop{\text{argmin}}_{#1}}
   \newcommand{\Ee}{\mathbb{E}}
   \newcommand{\Pp}{\mathbb{P}}
   \newcommand{\real}{\mathbb{R}}
   \newcommand{\Ybar}{\overline{Y}}
   \newcommand{\Yh}{\widehat{Y}}
   \newcommand{\Xbar}{\overline{X}}
   \newcommand{\Tr}{\text{Tr}}


   \newcommand{\model}{{\cal M}}

   \newcommand{\figvskip}{-0.7in}
   \newcommand{\fighskip}{-0.3in}
   \newcommand{\figheight}{3.5in}

   \newcommand{\Rcode}[1]{{\bf \tt #1 }}
   \newcommand{\Rtcode}[1]{{\tiny \bf \tt #1 }}
   \newcommand{\Rscode}[1]{{\small \bf \tt #1 }}

   \newcommand{\RR}{{\tt R} \;}
   \newcommand{\basename}[1]{http://stats60.stanford.edu/#1}
   \newcommand{\dataname}[1]{\basename{data/#1}}
   \newcommand{\Rname}[1]{\basename{R/#1}}

   \newcommand{\mycolor}[1]{{\color{blue} #1}}
   \newcommand{\basehref}[2]{\href{\basename{#1}}{\mycolor{#2}}}
   \newcommand{\Rhref}[2]{\href{\basename{R/#1}}{\mycolor{#2}}}
   \newcommand{\datahref}[2]{\href{\dataname{#1}}{\mycolor{#2}}}
   \newcommand{\X}{\pmb{X}}
   \newcommand{\Y}{\pmb{Y}}
   \newcommand{\be}{\pmb{varepsilon}}
   \newcommand{\logit}{\text{logit}}


   \title{Statistics 60: Introduction to Statistical Methods}
   \subtitle{Chapters 17 \& 18: Probability histograms, normal approximation} 
   \author{}% {Jonathan Taylor \\
   %Department of Statistics \\
   %Stanford University
   %}


   \begin{document}

   \begin{frame}
   \titlepage
   \end{frame}

   %%%%%%%%%%%%%%%%%%%%%%%%%%%%%%%%%%%%%%%%%%%%%%%%%%%%%%%%%%%%

   \begin{frame} \frametitle{Probability histogram}

   \begin{block}
     {Tossing a fair coin}
     \begin{itemize}
     \item When tossing a fair coin, there is 1/2 probability of getting
       1 head, 1/2 of getting 0 heads.

       \item We can make a histogram with an rectangle of width 1,
         area 1/2 around 0, and an identical rectangle around 1.
     \end{itemize}
   \end{block}
   \end{frame}

   %CODE
       % from matplotlib import rc
   % import pylab, numpy as np, sys
   % np.random.seed(0);import random; random.seed(0)
   % 
   % p = 1./2
   % x, p = conv_binom(p, 1)
   % cp = np.cumsum(p)
   % k = (cp >= 0)
   % pylab.bar(x[k],p[k], width=1, align='center', alpha=0.7)
   % a = pylab.gca()
   % a.set_ylim([0,0.6])
   % a.set_xlim([-0.6,1.6])
   % 


   \begin{frame}
   \frametitle{Probability histogram of successes}
   \begin{center}
   \resizebox{!}{2.7in}{\includegraphics{./images/inline/6c4aa4b4c4.pdf}}    
   \end{center}
   Tossing a fair coin 1 time, counting heads
   \end{frame}

   %%%%%%%%%%%%%%%%%%%%%%%%%%%%%%%%%%%%%%%%%%%%%%%%%%%%%%%%%%%%

   \begin{frame} \frametitle{Probability histogram}

   \begin{block}
     {Tossing a fair coin}
     \begin{itemize}
     \item When tossing a fair coin twice , there is
       \begin{itemize}
       \item 1/4 probability of getting 2 heads
       \item 1/2 probability of getting 1 head
       \item 1/4 probability of getting 0 heads
       \end{itemize}
       \item We can make similarly make a histogram for this experiment.
       \item This histogram is called a {\em probability histogram}.
     \end{itemize}
   \end{block}
   \end{frame}

   %CODE
       % from matplotlib import rc
   % import pylab, numpy as np, sys
   % np.random.seed(0);import random; random.seed(0)
   % 
   % p = 1./2
   % x, p = conv_binom(p, 2)
   % cp = np.cumsum(p)
   % k = (cp >= 0)
   % pylab.bar(x[k],p[k], width=1, align='center', alpha=0.7)
   % a = pylab.gca()
   % a.set_ylim([0,0.6])
   % a.set_xlim([-0.6,2.6])
   % 


   \begin{frame}
   \frametitle{Probability histogram of successes}
   \begin{center}
   \resizebox{!}{2.7in}{\includegraphics{./images/inline/29b86a221f.pdf}}    
   \end{center}
   Tossing a fair coin 2 times, counting heads
   \end{frame}

   %%%%%%%%%%%%%%%%%%%%%%%%%%%%%%%%%%%%%%%%%%%%%%%%%%%%%%%%%%%%

   \begin{frame} \frametitle{Probability histogram}

   \begin{block}
     {Probability histogram}
     \begin{itemize}
     \item Choose an experiment (e.g. tossing a fair coin twice and
     counting the number of heads, $H$).
     \item Repeat the experiment many times, say, 1000, creating a list
     $$
     [H_1, H_2, \dots, H_{1000}].
     $$
     \item The frequentist view of probability tells us that
     the histogram of the list $[H_1, H_2, \dots, H_{1000}]$
     should look like the {\em probability histogram}.

     \item Or, the empirical histogram {\em converges} to the
     probability histogram.
     \item We call this the {\em Law of Large Numbers}
     \end{itemize}
   \end{block}
   \end{frame}

   %CODE
       % from matplotlib import rc
   % import pylab, numpy as np, sys
   % np.random.seed(0);import random; random.seed(0)
   % 
   % X = np.random.binomial(2,0.5,1000)
   % p = np.array([(X == i).sum() for i in range(3)])/1000.
   % pylab.bar(np.arange(3),p, width=1, align='center', alpha=0.7,
   %           color='orange')
   % a = pylab.gca()
   % a.set_ylim([0,0.6])
   % a.set_xlim([-0.6,2.6])
   % 


   \begin{frame}
   \frametitle{Empirical histogram of successes}
   \begin{center}
   \resizebox{!}{2.7in}{\includegraphics{./images/inline/d2f3c3f2ce.pdf}}    
   \end{center}
   Tossing a fair coin 2 times, repeating 1000 times, counting heads
   \end{frame}

   %CODE
       % from matplotlib import rc
   % import pylab, numpy as np, sys
   % np.random.seed(0);import random; random.seed(0)
   % 
   % p = 1./2
   % x, p = conv_binom(p, 5)
   % cp = np.cumsum(p)
   % k = (cp >= 0)
   % pylab.bar(x[k],p[k], width=1, align='center', alpha=0.7)
   % 


   \begin{frame}
   \frametitle{Probability histogram of successes}
   \begin{center}
   \resizebox{!}{2.7in}{\includegraphics{./images/inline/9b704042df.pdf}}    
   \end{center}
   Tossing a fair coin 5 times, counting heads
   \end{frame}

   %CODE
       % from matplotlib import rc
   % import pylab, numpy as np, sys
   % np.random.seed(0);import random; random.seed(0)
   % 
   % X = np.random.binomial(5,0.5,1000)
   % p = np.array([(X == i).sum() for i in range(6)])/1000.
   % pylab.bar(np.arange(6),p, width=1, align='center', alpha=0.7,
   %           color='orange')
   % a = pylab.gca()
   % # a.set_ylim([0,0.6])
   % a.set_xlim([-0.6,5.6])
   % 


   \begin{frame}
   \frametitle{Empirical histogram of successes}
   \begin{center}
   \resizebox{!}{2.7in}{\includegraphics{./images/inline/7dfe81ed16.pdf}}    
   \end{center}
   Tossing a fair coin 5 times, repeating 1000 times.
   \end{frame}

   %CODE
       % from matplotlib import rc
   % import pylab, numpy as np, sys
   % np.random.seed(0);import random; random.seed(0)
   % 
   % p = 1./2
   % x, p = conv_binom(p, 100)
   % cp = np.cumsum(p)
   % k = (cp <= 0.999) * (cp >= 0.001)
   % pylab.bar(x[k],p[k], width=1, align='center', alpha=0.7)
   % 


   \begin{frame}
   \frametitle{Probability histogram of successes}
   \begin{center}
   \resizebox{!}{2.7in}{\includegraphics{./images/inline/b525f121bd.pdf}}    
   \end{center}
   Tossing a fair coin 100 times, counting heads
   \end{frame}

   %CODE
       % from matplotlib import rc
   % import pylab, numpy as np, sys
   % np.random.seed(0);import random; random.seed(0)
   % dx = 0.02
   % X, Y = np.mgrid[0.2:0.8:5j,0:1:8j]
   % X += np.random.uniform(0,1,X.shape) * dx - dx / 2
   % Y += np.random.uniform(0,1,Y.shape) * dx - dx / 2
   % ts = range(1,37) + ['0', '00']
   % success = [5]
   % tt = [('F', 'yellow')] * 38
   % for r in success:
   %     tt[r] = ('S', 'orange')
   % #np.random.shuffle(tt)
   % X.shape = -1; Y.shape = -1
   % g = np.array([t[1] == 'orange' for t in tt])
   % pylab.scatter(X[:38][g],Y[:38][g],s=500, c='orange', alpha=0.5)
   % pylab.scatter(X[:38][~g],Y[:38][~g],s=500, c='yellow', alpha=0.5)
   % for i, t in enumerate(tt):
   %     pylab.text(X[i], Y[i], t[0], color='black', ha='center', va='center')
   % 
   % pylab.gca().set_xticks([])
   % pylab.gca().set_yticks([])
   % pylab.gca().set_xlim([X.min()-0.1,X.max()+0.1])
   % pylab.gca().set_ylim([Y.min()-0.1,Y.max()+0.2])
   % 


   \begin{frame}
   \frametitle{Roulette}
   \begin{center}
   \resizebox{!}{2.7in}{\includegraphics{./images/inline/ecbeb53f65.pdf}}    
   \end{center}
   Betting on {\color{red} 5}: 1 ball is a success.
   \end{frame}

   %CODE
       % from matplotlib import rc
   % import pylab, numpy as np, sys
   % np.random.seed(0);import random; random.seed(0)
   % 
   % p = 1./38
   % x, p = conv_binom(p, 10)
   % cp = np.cumsum(p)
   % k = (cp >= 0)
   % pylab.bar(x[k],p[k], width=1, align='center', alpha=0.7)
   % 


   \begin{frame}
   \frametitle{Probability histogram of successes}
   \begin{center}
   \resizebox{!}{2.7in}{\includegraphics{./images/inline/2d9a93b253.pdf}}    
   \end{center}
   Betting on {\color{red} 5} 10 times, counting wins.
   \end{frame}

   %CODE
       % from matplotlib import rc
   % import pylab, numpy as np, sys
   % np.random.seed(0);import random; random.seed(0)
   % 
   % p = 1./38
   % 
   % x, p = conv_binom(p, 100)
   % cp = np.cumsum(p)
   % k = (cp <= 0.999) * (cp >= 0.001)
   % pylab.bar(x[k],p[k], width=1, align='center', alpha=0.7)
   % 


   \begin{frame}
   \frametitle{Probability histogram of successes}
   \begin{center}
   \resizebox{!}{2.7in}{\includegraphics{./images/inline/f4b44d207f.pdf}}    
   \end{center}
   Betting on {\color{red} 5} 100 times, counting wins.
   \end{frame}

   %CODE
       % from matplotlib import rc
   % import pylab, numpy as np, sys
   % np.random.seed(0);import random; random.seed(0)
   % 
   % p = 1./38
   % x, p = conv_binom(p, 1000)
   % cp = np.cumsum(p)
   % k = (cp <= 0.999) * (cp >= 0.001)
   % pylab.bar(x[k],p[k], width=1, align='center', alpha=0.7)
   % 


   \begin{frame}
   \frametitle{Probability histogram of successes}
   \begin{center}
   \resizebox{!}{2.7in}{\includegraphics{./images/inline/7cf4acaefe.pdf}}    
   \end{center}
   Betting on {\color{red} 5} 1000 times, counting wins.
   \end{frame}

   %CODE
       % from matplotlib import rc
   % import pylab, numpy as np, sys
   % np.random.seed(0);import random; random.seed(0)
   % 
   % p = 1./38
   % n = 10
   % x, p = conv_binom(p, n)
   % cp = np.cumsum(p)
   % k = (cp >= 0)
   % XX = np.zeros(x.shape[0]+1)
   % XX[:-1] = 100 + 350 * x - 10 * (n - x)
   % XX[-1] = XX[-2] + 100
   % PL_density(p, XX-170)
   % pylab.gca().set_xlim([-300,1200])
   % pylab.gca().set_ylabel('$\%$ per \$')
   % pylab.gca().set_xlabel('\$')
   % 


   \begin{frame}
   \frametitle{Probability histogram of successes}
   \begin{center}
   \resizebox{!}{2.7in}{\includegraphics{./images/inline/9de9c64fe1.pdf}}    
   \end{center}
   Betting on {\color{red} 5} 10 times, counting total balance.
   \end{frame}

   %CODE
       % from matplotlib import rc
   % import pylab, numpy as np, sys
   % np.random.seed(0);import random; random.seed(0)
   % 
   % p = 1./38
   % n = 100
   % x, p = conv_binom(p, n)
   % cp = np.cumsum(p)
   % k = (cp >= 0)
   % XX = np.zeros(x.shape[0]+1)
   % XX[:-1] = 100 + 350 * x - 10 * (n - x)
   % XX[-1] = XX[-2] + 100
   % PL_density(p, XX-170)
   % pylab.gca().set_ylabel('$\%$ per \$')
   % pylab.gca().set_xlim([-1000,3000])
   % pylab.gca().set_xlabel('\$')
   % 


   \begin{frame}
   \frametitle{Probability histogram of successes}
   \begin{center}
   \resizebox{!}{2.7in}{\includegraphics{./images/inline/ca234df1c9.pdf}}    
   \end{center}
   Betting on {\color{red} 5} 100 times, counting total balance.
   \end{frame}

   %CODE
       % from matplotlib import rc
   % import pylab, numpy as np, sys
   % np.random.seed(0);import random; random.seed(0)
   % 
   % p = 1./38
   % n = 1000
   % x, p = conv_binom(p, n)
   % cp = np.cumsum(p)
   % k = (cp >= 0)
   % XX = np.zeros(x.shape[0]+1)
   % XX[:-1] = 100 + 350 * x - 10 * (n - x)
   % XX[-1] = XX[-2] + 100
   % PL_density(p, XX-170)
   % pylab.gca().set_ylabel('$\%$ per \$')
   % pylab.gca().set_xlim([-6000,6000])
   % pylab.gca().set_xlabel('\$')
   % 


   \begin{frame}
   \frametitle{Probability histogram of successes}
   \begin{center}
   \resizebox{!}{2.7in}{\includegraphics{./images/inline/6d19392a52.pdf}}    
   \end{center}
   Betting on {\color{red} 5} 1000 times, counting total money
   \end{frame}

   %%%%%%%%%%%%%%%%%%%%%%%%%%%%%%%%%%%%%%%%%%%%%%%%%%%%%%%%%%%%

   \begin{frame} \frametitle{Normal approximation}

   \begin{block}
   {Central limit theorem}
   \begin{itemize}
   \item When making many independent draws from a box, the central
   limit theorem says that we can use the normal curve to approximate
   probabilities of things for the {\bf sum of draws}.

   \item Specifically, the normal curve applies to
     $$
     \frac{\text{{\bf sum of draws}} - \text{average({\bf sum of draws})}}{\text{SE({\bf sum of draws})}}
     $$
   \end{itemize}
   \end{block}
   \end{frame}

   %%%%%%%%%%%%%%%%%%%%%%%%%%%%%%%%%%%%%%%%%%%%%%%%%%%%%%%%%%%%

   \begin{frame} \frametitle{Normal approximation}

   \begin{block}
   {Example}
   \begin{description}
   \item[Q] Roulette, betting on {\color{red} 5} 10 times, 10\$ each bet starting with 100 \$.
     What are the chances we will finish with more than 200 \$?
   \item[A] We know
     \begin{itemize}
     \item $\text{average({\bf sum of draws})} = 100 \times (-0.52)\$ = -52\$ $
     \item $\text{SE({\bf sum of draws})} = \sqrt{100} \times 360 \times \sqrt{\frac{1}{38} \times \frac{37}{38}} \approx 570\$ $
     \item Finishing with more than 200\$ means the {\bf sum of draws} was greater than 100\$ .
       \item In standardized units, this is
       $$
       \frac{100-(-52)}{570} \approx 0.27
       $$
     \end{itemize}
   \end{description}
   \end{block}
   \end{frame}

   %CODE
       % from matplotlib import rc
   % import pylab, numpy as np, sys
   % np.random.seed(0);import random; random.seed(0)
   % import pylab, numpy as np
   % x = np.linspace(-4,4,101)
   % y = np.exp(-x**2/2) / np.sqrt(2*np.pi)
   % x2 = np.linspace(0.264,4,101)
   % y2 = np.exp(-x2**2/2) / np.sqrt(2*np.pi)
   % pylab.plot(x,y*100, linewidth=2)
   % xf, yf = pylab.poly_between(x2, 0*x2, y2*100)
   % pylab.fill(xf, yf, facecolor='red', hatch='\\', alpha=0.5)
   % pylab.gca().set_xlabel('standardized units')
   % pylab.gca().set_ylabel('% per standardized unit')
   % #pylab.gca().set_xlim([-2,4])
   % #pylab.gca().set_yticks([])
   % 


   \begin{frame}
   \frametitle{Normal approximation}
   \begin{center}
   \resizebox{!}{2.7in}{\includegraphics{./images/inline/196e67e4c7.pdf}}    
   \end{center}
   From A-104, chances are approximately 40\%
   \end{frame}

   %CODE
       % from matplotlib import rc
   % import pylab, numpy as np, sys
   % np.random.seed(0);import random; random.seed(0)
   % 
   % p = 1./38
   % n = 100
   % x, p = conv_binom(p, n)
   % cp = np.cumsum(p)
   % k = (cp >= 0)
   % XX = np.zeros(x.shape[0]+1)
   % XX[:-1] = 100 + 350 * x - 10 * (n - x)
   % XX[-1] = XX[-2] + 1000
   % XY = XX-(XX[3]-XX[2])/2
   % PL_density(p, XY)
   % pylab.gca().set_ylabel('$\%$ per \$')
   % pylab.gca().set_xlim([-6000,6000])
   % pylab.gca().set_xlabel('\$')
   % 
   % mu = 100 + n * np.mean([-10]*37+[350])
   % SD = np.sqrt(n) * np.std([-10]*37+[350])
   % from scipy.stats import norm
   % XZ = np.linspace(XX.min(), XX.max(), 2000)
   % pylab.plot(XZ, 100 * norm.pdf((XZ-mu)/SD) / SD, linewidth=4, color='red')
   % pylab.gca().set_xlim([-6000,6000])
   % 


   \begin{frame}
   \frametitle{Probability histogram of successes}
   \begin{center}
   \resizebox{!}{2.7in}{\includegraphics{./images/inline/29a71def68.pdf}}    
   \end{center}
   Normal approximation using average, SE of sum of 100 draws
   \end{frame}

   %CODE
       % from matplotlib import rc
   % import pylab, numpy as np, sys
   % np.random.seed(0);import random; random.seed(0)
   % 
   % p = 1./38
   % n = 1000
   % x, p = conv_binom(p, n)
   % cp = np.cumsum(p)
   % k = (cp >= 0)
   % XX = np.zeros(x.shape[0]+1)
   % XX[:-1] = 100 + 350 * x - 10 * (n - x)
   % XX[-1] = XX[-2] + 1000
   % XY = XX-(XX[3]-XX[2])/2
   % PL_density(p, XY)
   % pylab.gca().set_ylabel('$\%$ per \$')
   % pylab.gca().set_xlim([-6000,6000])
   % pylab.gca().set_xlabel('\$')
   % 
   % mu = 100 + n * np.mean([-10]*37+[350])
   % SD = np.sqrt(n) * np.std([-10]*37+[350])
   % from scipy.stats import norm
   % pylab.plot(XX, 100 * norm.pdf((XX-mu)/SD) / SD, linewidth=4, color='red')
   % pylab.gca().set_xlim([-6000,6000])
   % 


   \begin{frame}
   \frametitle{Probability histogram of successes}
   \begin{center}
   \resizebox{!}{2.7in}{\includegraphics{./images/inline/ba0cd41e38.pdf}}    
   \end{center}
   Normal approximation using average, SE of sum of 1000 draws
   \end{frame}

   %CODE
       % from matplotlib import rc
   % import pylab, numpy as np, sys
   % np.random.seed(0);import random; random.seed(0)
   % 
   % p = 1./2
   % n = 100
   % x, p = conv_binom(p, n)
   % cp = np.cumsum(p)
   % k = (cp >= 0)
   % XX = np.zeros(x.shape[0]+1)
   % XX[:-1] = x
   % XX[-1] = XX[-2] + 10
   % XY = XX-(XX[3]-XX[2])/2.
   % PL_density(p, XY)
   % pylab.gca().set_ylabel('$\%$ per head')
   % pylab.gca().set_xlabel('# heads')
   % 
   % mu = n * np.mean([1,0])
   % SD = np.sqrt(n) * np.std([0,1])
   % from scipy.stats import norm
   % pylab.plot(XX, 100 * norm.pdf((XX-mu)/SD) / SD, linewidth=4, color='red')
   % 
   % pylab.gca().set_xlim([35,65])
   % 


   \begin{frame}
   \frametitle{Probability histogram of successes}
   \begin{center}
   \resizebox{!}{2.7in}{\includegraphics{./images/inline/2e823401b6.pdf}}    
   \end{center}
   Normal approximation of number of heads in 100 flips
   \end{frame}

   %CODE
       % from matplotlib import rc
   % import pylab, numpy as np, sys
   % np.random.seed(0);import random; random.seed(0)
   % 
   % p = 1./2
   % n = 400
   % x, p = conv_binom(p, n)
   % cp = np.cumsum(p)
   % k = (cp >= 0)
   % XX = np.zeros(x.shape[0]+1)
   % XX[:-1] = x
   % XX[-1] = XX[-2] + 10
   % XY = XX-(XX[3]-XX[2])/2.
   % PL_density(p, XY)
   % pylab.gca().set_ylabel('$\%$ per head')
   % pylab.gca().set_xlabel('# heads')
   % 
   % mu = n * np.mean([1,0])
   % SD = np.sqrt(n) * np.std([0,1])
   % from scipy.stats import norm
   % pylab.plot(XX, 100 * norm.pdf((XX-mu)/SD) / SD, linewidth=4, color='red')
   % 
   % pylab.gca().set_xlim([170,230])
   % 


   \begin{frame}
   \frametitle{Probability histogram of successes}
   \begin{center}
   \resizebox{!}{2.7in}{\includegraphics{./images/inline/8fb928ab07.pdf}}    
   \end{center}
   Normal approximation of number of heads in 400 flips
   \end{frame}

   %CODE
       % from matplotlib import rc
   % import pylab, numpy as np, sys
   % np.random.seed(0);import random; random.seed(0)
   % 
   % p = 1./2
   % n = 900
   % x, p = conv_binom(p, n)
   % cp = np.cumsum(p)
   % k = (cp >= 0)
   % XX = np.zeros(x.shape[0]+1)
   % XX[:-1] = x
   % XX[-1] = XX[-2] + 10
   % XY = XX-(XX[3]-XX[2])/2.
   % PL_density(p, XY)
   % pylab.gca().set_ylabel('$\%$ per head')
   % pylab.gca().set_xlabel('# heads')
   % 
   % mu = n * np.mean([1,0])
   % SD = np.sqrt(n) * np.std([0,1])
   % from scipy.stats import norm
   % pylab.plot(XX, 100 * norm.pdf((XX-mu)/SD) / SD, linewidth=4, color='red')
   % 
   % pylab.gca().set_xlim([405,495])
   % 


   \begin{frame}
   \frametitle{Probability histogram of successes}
   \begin{center}
   \resizebox{!}{2.7in}{\includegraphics{./images/inline/8b2908b26c.pdf}}    
   \end{center}
   Normal approximation of number of heads in 400 flips
   \end{frame}

   %%%%%%%%%%%%%%%%%%%%%%%%%%%%%%%%%%%%%%%%%%%%%%%%%%%%%%%%%%%%

   \begin{frame} \frametitle{Normal approximation}

   \begin{block}
   {Continuity correction}
   \begin{itemize}
   \item When using the normal approximation to the sum of draws,
     we sometimes use the {\em continuity correction}.
     \begin{itemize}
     \item \{observing more than 40 heads in 100 flips\} =
     \{observing more than 40.5 heads in 100 flips\}

     \item \{observing less than 40 heads in 100 flips\}
      = \{observing less than 39.5 heads in 100 flips\}

      \item \{observing exactly 40 heads in 100 flips\} =
       \{observing between 39.5 and 40.5 heads in 100 flips\}
      \item \{observing greater than or equal to 41 heads but less
       than 52 heads in 100 flips \} = \{observing between 40.5 and 51.5 heads in 100 flips\}
     \end{itemize}

     \end{itemize}

   \end{block}
   \end{frame}

   %CODE
       % from matplotlib import rc
   % import pylab, numpy as np, sys
   % np.random.seed(0);import random; random.seed(0)
   % 
   % from scipy.stats import norm
   % 
   % p = 1./2
   % n = 100
   % x, p = conv_binom(p, n)
   % cp = np.cumsum(p)
   % k = (cp >= 0)
   % XX = np.zeros(x.shape[0]+1)
   % XX[:-1] = x
   % XX[-1] = XX[-2] + 10
   % XY = XX-(XX[3]-XX[2])/2.
   % PL_density(p, XY)
   % pylab.gca().set_ylabel('$\%$ per head')
   % pylab.gca().set_xlabel('# heads')
   % 
   % mu = n * np.mean([1,0])
   % SD = np.sqrt(n) * np.std([0,1])
   % pylab.plot(XX, 100 * norm.pdf((XX-mu)/SD) / SD, linewidth=4, color='red')
   % 
   % x2 = np.linspace(40.5,100,1001)
   % y2 = 100 * norm.pdf((x2-mu)/SD) / SD
   % xf, yf = pylab.poly_between(x2, 0*x2, y2)
   % pylab.fill(xf, yf, facecolor='red', hatch='\\', alpha=0.7)
   % 
   % 
   % pylab.gca().set_xlim([35,65])
   % 


   \begin{frame}
   \frametitle{Normal approximation}
   \begin{center}
   \resizebox{!}{2.7in}{\includegraphics{./images/inline/bd2cdcdef4.pdf}}    
   \end{center}
   Observing more than 40 heads using continuity correction
   \end{frame}

   %CODE
       % from matplotlib import rc
   % import pylab, numpy as np, sys
   % np.random.seed(0);import random; random.seed(0)
   % 
   % from scipy.stats import norm
   % 
   % p = 1./2
   % n = 100
   % x, p = conv_binom(p, n)
   % cp = np.cumsum(p)
   % k = (cp >= 0)
   % XX = np.zeros(x.shape[0]+1)
   % XX[:-1] = x
   % XX[-1] = XX[-2] + 10
   % XY = XX-(XX[3]-XX[2])/2.
   % PL_density(p, XY)
   % pylab.gca().set_ylabel('$\%$ per head')
   % pylab.gca().set_xlabel('# heads')
   % 
   % mu = n * np.mean([1,0])
   % SD = np.sqrt(n) * np.std([0,1])
   % pylab.plot(XX, 100 * norm.pdf((XX-mu)/SD) / SD, linewidth=4, color='red')
   % 
   % x2 = np.linspace(0,39.5,1001)
   % y2 = 100 * norm.pdf((x2-mu)/SD) / SD
   % xf, yf = pylab.poly_between(x2, 0*x2, y2)
   % pylab.fill(xf, yf, facecolor='red', hatch='\\', alpha=0.7)
   % 
   % 
   % pylab.gca().set_xlim([35,65])
   % 


   \begin{frame}
   \frametitle{Normal approximation}
   \begin{center}
   \resizebox{!}{2.7in}{\includegraphics{./images/inline/6af52b6a07.pdf}}    
   \end{center}
   Observing less than 40 heads using continuity correction
   \end{frame}

   %CODE
       % from matplotlib import rc
   % import pylab, numpy as np, sys
   % np.random.seed(0);import random; random.seed(0)
   % 
   % from scipy.stats import norm
   % 
   % p = 1./2
   % n = 100
   % x, p = conv_binom(p, n)
   % cp = np.cumsum(p)
   % k = (cp >= 0)
   % XX = np.zeros(x.shape[0]+1)
   % XX[:-1] = x
   % XX[-1] = XX[-2] + 10
   % XY = XX-(XX[3]-XX[2])/2.
   % PL_density(p, XY)
   % pylab.gca().set_ylabel('$\%$ per head')
   % pylab.gca().set_xlabel('# heads')
   % 
   % mu = n * np.mean([1,0])
   % SD = np.sqrt(n) * np.std([0,1])
   % pylab.plot(XX, 100 * norm.pdf((XX-mu)/SD) / SD, linewidth=4, color='red')
   % 
   % x2 = np.linspace(39.5,40.5,101)
   % y2 = 100 * norm.pdf((x2-mu)/SD) / SD
   % xf, yf = pylab.poly_between(x2, 0*x2, y2)
   % pylab.fill(xf, yf, facecolor='red', hatch='\\', alpha=0.7)
   % 
   % 
   % pylab.gca().set_xlim([35,65])
   % 


   \begin{frame}
   \frametitle{Normal approximation}
   \begin{center}
   \resizebox{!}{2.7in}{\includegraphics{./images/inline/6cfecf7d7e.pdf}}    
   \end{center}
   Observing exactly 40 heads using continuity correction
   \end{frame}

   %CODE
       % from matplotlib import rc
   % import pylab, numpy as np, sys
   % np.random.seed(0);import random; random.seed(0)
   % 
   % from scipy.stats import norm
   % 
   % p = 1./2
   % n = 100
   % x, p = conv_binom(p, n)
   % cp = np.cumsum(p)
   % k = (cp >= 0)
   % XX = np.zeros(x.shape[0]+1)
   % XX[:-1] = x
   % XX[-1] = XX[-2] + 10
   % XY = XX-(XX[3]-XX[2])/2.
   % PL_density(p, XY)
   % pylab.gca().set_ylabel('$\%$ per head')
   % pylab.gca().set_xlabel('# heads')
   % 
   % mu = n * np.mean([1,0])
   % SD = np.sqrt(n) * np.std([0,1])
   % pylab.plot(XX, 100 * norm.pdf((XX-mu)/SD) / SD, linewidth=4, color='red')
   % 
   % x2 = np.linspace(40.5,51.5,101)
   % y2 = 100 * norm.pdf((x2-mu)/SD) / SD
   % xf, yf = pylab.poly_between(x2, 0*x2, y2)
   % pylab.fill(xf, yf, facecolor='red', hatch='\\', alpha=0.7)
   % 
   % 
   % pylab.gca().set_xlim([35,65])
   % 


   \begin{frame}
   \frametitle{Normal approximation}
   \begin{center}
   \resizebox{!}{2.7in}{\includegraphics{./images/inline/f7e8cc8980.pdf}}    
   \end{center}
   Observing greater than or equal to 41 but less than 52.
   \end{frame}

   %%%%%%%%%%%%%%%%%%%%%%%%%%%%%%%%%%%%%%%%%%%%%%%%%%%%%%%%%%%%

   \begin{frame} \frametitle{Normal approximation}

   \begin{block}
   {Example}
   \begin{description}
   \item[Q] Use the normal approximation to estimate the
     probability of observing more than or equal to 45 heads in 80 flips of a fair coin.
   \item[A] We know
     \begin{itemize}
     \item $\text{average({\bf sum of draws})} = 80 \times 0.5 = 40 $
     \item $\text{SE({\bf sum of draws})} = \sqrt{80} \times \sqrt{\frac{1}{2}  \times \frac{1}{2}} \approx 4.5 $
     \item Observing more than or equal to 45 heads is the same as observing more than 44.5 heads
       \item In standardized units, this is
       $$
       \frac{44.5-40}{4.5} \approx 1
       $$
     \end{itemize}
   \end{description}
   \end{block}
   \end{frame}

   %CODE
       % from matplotlib import rc
   % import pylab, numpy as np, sys
   % np.random.seed(0);import random; random.seed(0)
   % import pylab, numpy as np
   % x = np.linspace(-4,4,101)
   % y = np.exp(-x**2/2) / np.sqrt(2*np.pi)
   % x2 = np.linspace(1,4,101)
   % y2 = np.exp(-x2**2/2) / np.sqrt(2*np.pi)
   % pylab.plot(x,y*100, linewidth=2)
   % xf, yf = pylab.poly_between(x2, 0*x2, y2*100)
   % pylab.fill(xf, yf, facecolor='red', hatch='\\', alpha=0.5)
   % pylab.gca().set_xlabel('standardized units')
   % pylab.gca().set_ylabel('% per standardized unit')
   % #pylab.gca().set_xlim([-2,4])
   % #pylab.gca().set_yticks([])
   % 


   \begin{frame}
   \frametitle{Normal approximation}
   \begin{center}
   \resizebox{!}{2.7in}{\includegraphics{./images/inline/a528a70bde.pdf}}    
   \end{center}
   From A-104, chances are approximately 16\%
   \end{frame}

   %%%%%%%%%%%%%%%%%%%%%%%%%%%%%%%%%%%%%%%%%%%%%%%%%%%%%%%%%%%%

   \begin{frame} \frametitle{Normal approximation}

   \begin{block}
   {Example (continued)}
   \begin{description}
   \item[Q] Use the normal approximation to estimate the
     probability of observing exactly 45 heads in 80 flips of a fair coin.
   \item[A] We know
     \begin{itemize}
     \item $\text{average({\bf sum of draws})} = 80 \times 0.5 = 40 $
     \item $\text{SE({\bf sum of draws})} = \sqrt{80} \times \sqrt{\frac{1}{2} \times  \frac{1}{2}} \approx 4.5 $
     \item Observing 45 heads is the same as observing between 44.5 and 45.5 heads.
       \item In standardized units, this is endpoints are
       $$
       \begin{aligned}
       \text{lower}&= \frac{44.5-40}{4.5}    \approx 1 \\
       \text{upper}&= \frac{45.5-40}{4.5}    \approx 1.2 \\
       \end{aligned}
       $$
     \end{itemize}
   \end{description}
   \end{block}
   \end{frame}

   %CODE
       % from matplotlib import rc
   % import pylab, numpy as np, sys
   % np.random.seed(0);import random; random.seed(0)
   % import pylab, numpy as np
   % x = np.linspace(-4,4,101)
   % y = np.exp(-x**2/2) / np.sqrt(2*np.pi)
   % x2 = np.linspace(1,1.2,101)
   % y2 = np.exp(-x2**2/2) / np.sqrt(2*np.pi)
   % pylab.plot(x,y*100, linewidth=2)
   % xf, yf = pylab.poly_between(x2, 0*x2, y2*100)
   % pylab.fill(xf, yf, facecolor='red', hatch='\\', alpha=0.5)
   % pylab.gca().set_xlabel('standardized units')
   % pylab.gca().set_ylabel('% per standardized unit')
   % #pylab.gca().set_xlim([-2,4])
   % #pylab.gca().set_yticks([])
   % 


   \begin{frame}
   \frametitle{Normal approximation}
   \begin{center}
   \resizebox{!}{2.7in}{\includegraphics{./images/inline/ec76a56c45.pdf}}    
   \end{center}
   From A-104: approximately 4.8\%, (actual 4.8\%)
   \end{frame}

   %%%%%%%%%%%%%%%%%%%%%%%%%%%%%%%%%%%%%%%%%%%%%%%%%%%%%%%%%%%%

   \begin{frame} \frametitle{Normal approximation}

   \begin{block}
   {Central limit theorem}
   \begin{itemize}
   \item The central limit theorem applies to {\bf sum of draws}.
   \item The number of draws should be reasonably large.
   \item The more lopsided the values are, the more draws needed for
   reasonable approximation (compare the approximations of rolling {\color{red} 5} in roulette
   to flipping a fair coin).
   \item It is another type of {\em convergence}: as the number
   of draws grows, the approximation gets better.
   \end{itemize}
   \end{block}
   \end{frame}

   %CODE
       % from matplotlib import rc
   % import pylab, numpy as np, sys
   % np.random.seed(0);import random; random.seed(0)
   % from scipy.stats import norm
   % 
   % d = np.array([0,1,1.,0,0,0,0,1])/3
   % n = 10
   % x, d = conv_integer_rv(d, n)
   % cp = np.cumsum(d)
   % k = (cp >= 0.001) * (cp <= 0.999)
   % 
   % pylab.bar(x[k],d[k] * 100, width=1, align='center', alpha=0.7)
   % mu = n * np.mean([1,2,7])
   % SD = np.sqrt(n) * np.std([1,2,7])
   % pylab.plot(x[k],norm.pdf((x[k]-mu)/SD) / SD * 100, linewidth=4, color='red')
   % 


   \begin{frame}
   \frametitle{Normal approximation}
   \begin{center}
   \resizebox{!}{2.7in}{\includegraphics{./images/inline/a820e815e0.pdf}}    
   \end{center}
   10 draws from a lopsided box: [1,2,7]
   \end{frame}

   %CODE
       % from matplotlib import rc
   % import pylab, numpy as np, sys
   % np.random.seed(0);import random; random.seed(0)
   % from scipy.stats import norm
   % 
   % d = np.array([0,1,1.,0,0,0,0,1])/3
   % n = 30
   % x, d = conv_integer_rv(d, n)
   % cp = np.cumsum(d)
   % k = (cp >= 0.001) * (cp <= 0.999)
   % 
   % pylab.bar(x[k],d[k] * 100, width=1, align='center', alpha=0.7)
   % mu = n * np.mean([1,2,7])
   % SD = np.sqrt(n) * np.std([1,2,7])
   % pylab.plot(x[k],norm.pdf((x[k]-mu)/SD) / SD * 100, linewidth=4, color='red')
   % 


   \begin{frame}
   \frametitle{Normal approximation}
   \begin{center}
   \resizebox{!}{2.7in}{\includegraphics{./images/inline/8d90ec25bf.pdf}}    
   \end{center}
   30 draws from a lopsided box: [1,2,7]
   \end{frame}

   %CODE
       % from matplotlib import rc
   % import pylab, numpy as np, sys
   % np.random.seed(0);import random; random.seed(0)
   % from scipy.stats import norm
   % 
   % d = np.array([0,1,1.,0,0,0,0,1])/3
   % n = 50
   % x, d = conv_integer_rv(d, n)
   % cp = np.cumsum(d)
   % k = (cp >= 0.001) * (cp <= 0.999)
   % 
   % pylab.bar(x[k],d[k] * 100, width=1, align='center', alpha=0.7)
   % mu = n * np.mean([1,2,7])
   % SD = np.sqrt(n) * np.std([1,2,7])
   % pylab.plot(x[k],norm.pdf((x[k]-mu)/SD) / SD * 100, linewidth=4, color='red')
   % 


   \begin{frame}
   \frametitle{Normal approximation}
   \begin{center}
   \resizebox{!}{2.7in}{\includegraphics{./images/inline/5ff3d2b828.pdf}}    
   \end{center}
   50 draws from a lopsided box: [1,2,7]
   \end{frame}

   %%%%%%%%%%%%%%%%%%%%%%%%%%%%%%%%%%%%%%%%%%%%%%%%%%%%%%%%%%%%

   \begin{frame} 

   \end{frame}

   %%%%%%%%%%%%%%%%%%%%%%%%%%%%%%%%%%%%%%%%%%%%%%%%%%%%%%%%%%%%

   \begin{frame} 

   \end{frame}

   \end{document}
