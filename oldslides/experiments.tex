   \documentclass[handout]{beamer}



   \mode<presentation>
   {
     \usetheme{PaloAlto}
   \setbeamertemplate{footline}[page number]

     \setbeamercolor{sidebar}{bg=white, fg=black}
     \setbeamercolor{frametitle}{bg=white, fg=black}
     % or ...
     \setbeamercolor{logo}{bg=white}
     \setbeamercolor{block body}{parent=normal text,bg=white}
     \setbeamercolor{author in sidebar}{fg=black}
     \setbeamercolor{title in sidebar}{fg=black}


     \setbeamercolor*{block title}{use=structure,fg=structure.fg,bg=structure.fg!20!bg}
     \setbeamercolor*{block title alerted}{use=alerted text,fg=alerted text.fg,bg=alerted text.fg!20!bg}
     \setbeamercolor*{block title example}{use=example text,fg=example text.fg,bg=example text.fg!20!bg}


     \setbeamercolor{block body}{parent=normal text,use=block title,bg=block title.bg!50!bg}
     \setbeamercolor{block body alerted}{parent=normal text,use=block title alerted,bg=block title alerted.bg!50!bg}
     \setbeamercolor{block body example}{parent=normal text,use=block title example,bg=block title example.bg!50!bg}

     % or ...

     \setbeamercovered{transparent}
     % or whatever (possibly just delete it)
     \logo{\resizebox{!}{1.5cm}{\href{\basename{R}}{\includegraphics{image}}}}
   }

   \mode<handout>
   {
     \usetheme{PaloAlto}
     \usecolortheme{default}
     \setbeamercolor{sidebar}{bg=white, fg=black}
     \setbeamercolor{frametitle}{bg=white, fg=black}
     % or ...
     \setbeamercolor{logo}{bg=white}
     \setbeamercolor{block body}{parent=normal text,bg=white}
     \setbeamercolor{author in sidebar}{fg=black}
     \setbeamercolor{title in sidebar}{fg=black}
     \setbeamercovered{transparent}
     % or whatever (possibly just delete it)
     \logo{}
   }

   \usepackage{epsdice}
   \usepackage[latin1]{inputenc}
   \usepackage{graphics}
   \usepackage{amsmath,eepic,epic}

   \newcommand{\figslide}[3]{
   \begin{frame}
   \frametitle{#1}
     \begin{center}
     \resizebox{!}{2.7in}{\includegraphics{#2}}    
     \end{center}
   {#3}
   \end{frame}
   }

   \newcommand{\fighslide}[4]{
   \begin{frame}
   \frametitle{#1}
     \begin{center}
     \resizebox{!}{#4}{\includegraphics{#2}}    
     \end{center}
   {#3}
   \end{frame}
   }

   \newcommand{\figwref}[1]{
   \href{#1}{\tiny \tt #1}}

   \newcommand{\B}[1]{\beta_{#1}}
   \newcommand{\Bh}[1]{\widehat{\beta}_{#1}}
   \newcommand{\V}{\text{Var}}
   \newcommand{\Cov}{\text{Cov}}
   \newcommand{\Vh}{\widehat{\V}}
   \newcommand{\s}{\sigma}
   \newcommand{\sh}{\widehat{\sigma}}

   \newcommand{\argmax}[1]{\mathop{\text{argmax}}_{#1}}
   \newcommand{\argmin}[1]{\mathop{\text{argmin}}_{#1}}
   \newcommand{\Ee}{\mathbb{E}}
   \newcommand{\Pp}{\mathbb{P}}
   \newcommand{\real}{\mathbb{R}}
   \newcommand{\Ybar}{\overline{Y}}
   \newcommand{\Yh}{\widehat{Y}}
   \newcommand{\Xbar}{\overline{X}}
   \newcommand{\Tr}{\text{Tr}}


   \newcommand{\model}{{\cal M}}

   \newcommand{\figvskip}{-0.7in}
   \newcommand{\fighskip}{-0.3in}
   \newcommand{\figheight}{3.5in}

   \newcommand{\Rcode}[1]{{\bf \tt #1 }}
   \newcommand{\Rtcode}[1]{{\tiny \bf \tt #1 }}
   \newcommand{\Rscode}[1]{{\small \bf \tt #1 }}

   \newcommand{\RR}{{\tt R} \;}
   \newcommand{\basename}[1]{http://stats60.stanford.edu/#1}
   \newcommand{\dataname}[1]{\basename{data/#1}}
   \newcommand{\Rname}[1]{\basename{R/#1}}

   \newcommand{\mycolor}[1]{{\color{blue} #1}}
   \newcommand{\basehref}[2]{\href{\basename{#1}}{\mycolor{#2}}}
   \newcommand{\Rhref}[2]{\href{\basename{R/#1}}{\mycolor{#2}}}
   \newcommand{\datahref}[2]{\href{\dataname{#1}}{\mycolor{#2}}}
   \newcommand{\X}{\pmb{X}}
   \newcommand{\Y}{\pmb{Y}}
   \newcommand{\be}{\pmb{varepsilon}}
   \newcommand{\logit}{\text{logit}}


   \title{Statistics 60: Introduction to Statistical Methods}
   \subtitle{Chapters 1 \& 2: Controlled Experiments \& Observation Studies} 
   \author{}% {Jonathan Taylor \\
   %Department of Statistics \\
   %Stanford University
   %}


   \begin{document}

   \begin{frame}
   \titlepage
   \end{frame}

   %%%%%%%%%%%%%%%%%%%%%%%%%%%%%%%%%%%%%%%%%%%%%%%%%%%%%%%%%%%%

   \begin{frame} \frametitle{Experiments}

   \begin{block}
   {Prototypical examples}
   \begin{itemize}
   \item Does smoking cause lung cancer?
   \item How can this be established -- by comparing smokers to non-smokers?
   \item Is a vaccine effective against some infectious disease?
   \end{itemize}
   \end{block}
   \end{frame}

   %%%%%%%%%%%%%%%%%%%%%%%%%%%%%%%%%%%%%%%%%%%%%%%%%%%%%%%%%%%%

   \begin{frame} \frametitle{Experiments}

   \begin{block}
   {Treatment and control groups}

   \begin{itemize}
   \item In the smoking study: {\bf \color{red} treatment = smokers}, {\bf \color{blue} control = non-smokers}.
   \item In a  vaccine study: {\bf \color{red} treatment = patients who are vaccinated}, {\bf \color{blue} control = non-vaccinated}.
   \item Ideally, the only difference between {\bf \color{red} treatment} and {\bf \color{blue} control} is whether or not they receive the treatment.
   \end{itemize}
   \end{block}

   \begin{block}
   {Randomized controlled experiments}

   \begin{itemize}
   \item The best way to establish smoking causes lung cancer is a {\em randomized controlled} experiment.
   \item In such a study, patients would be assigned randomly to smoking or non-smoking group.
   \item Obviously, these experiments are not always possible: we can't force people to smoke.
   \end{itemize}
   \end{block}
   \end{frame}

   %%%%%%%%%%%%%%%%%%%%%%%%%%%%%%%%%%%%%%%%%%%%%%%%%%%%%%%%%%%%

   \begin{frame} \frametitle{Randomized Controlled Experiments}

   \begin{block}
   {Why randomize?}

   \begin{itemize}
   \item In the polio vaccine example described in the book, wealthy families whose children were more vulnerable to polio also were more likely to volunteer for vaccination.
   \item If {\bf \color{red} treatment} is assigned based on whoever volunteers, this biases the experiment against the vaccine, i.e. its apparent effectiveness is diminished.
   \item This means there will be differences between {\bf \color{red} treatment} and {\bf \color{blue} control} groups other than just the vaccine.
   \end{itemize}

   \end{block}
   \end{frame}

   %%%%%%%%%%%%%%%%%%%%%%%%%%%%%%%%%%%%%%%%%%%%%%%%%%%%%%%%%%%%

   \begin{frame} \frametitle{Randomized Controlled Experiments}

   \begin{block}
   {Placebo effect: blinding}
   \begin{itemize}
   \item    Subjects in the {\bf \color{blue} control} group should be given a ``treatment'' with no effect. That is, they should be {\em blinded}.
   \item Why? So the response is not due to the idea of a vaccine, but the
   vaccine itself.
   \item In the vaccination example, children were given an injection of salt and water.
   \item This treatment is called a {\bf \color{blue} placebo}.
   \end{itemize}

   \end{block}
   \end{frame}

   %%%%%%%%%%%%%%%%%%%%%%%%%%%%%%%%%%%%%%%%%%%%%%%%%%%%%%%%%%%%

   \begin{frame} \frametitle{Randomized Controlled Experiments}

   \begin{block}
   {Why randomize?}

   \begin{itemize}
   \item In the polio vaccine example described in the book, wealthy families whose children were more vulnerable to polio also were more likely to volunteer for vaccination.
   \item If treatment is assigned based on whoever volunteers, this biases the experiment against the vaccine, i.e. its apparent effectiveness is diminished.
   \end{itemize}

   \end{block}
   \end{frame}

   %%%%%%%%%%%%%%%%%%%%%%%%%%%%%%%%%%%%%%%%%%%%%%%%%%%%%%%%%%%%

   \begin{frame} \frametitle{Randomized Controlled Experiments}

   \begin{block}
   {Double blinding}
   \begin{itemize}
   \item  If the doctors know who receives treatment and who receives placebo, they may also bias the results by their reporting.
   \item For example, polio diagnosis is not perfect, a doctor with interest in the success of the vaccine may declare a treated child with mild polio as healthy; or an untreated (placebo) child who is close to healthy as having mild polio.
   \item This bias may be conscious or unconscious on the doctors' part.
   \end{itemize}

   \end{block}
   \end{frame}

   %%%%%%%%%%%%%%%%%%%%%%%%%%%%%%%%%%%%%%%%%%%%%%%%%%%%%%%%%%%%

   \begin{frame} \frametitle{Randomized controlled experiments}

   \begin{block}
   {NFIP study}
   \begin{tabular}{lrr}
   Group & Size & Rate \\ \hline
   Treatment & 225,000 & 25 \\
   Control & 725,000 & 54 \\
   No consent & 125,000 & 44
   \end{tabular}
   \end{block}


   \begin{block}
   {Double-blind study}
   \begin{tabular}{lrr}
   Group & Size & Rate \\ \hline
   Treatment & 200,000 & 28 \\
   Control & 200,000 & 71 \\
   No consent & 350,000 & 46
   \end{tabular}
   \end{block}
   \end{frame}

   %%%%%%%%%%%%%%%%%%%%%%%%%%%%%%%%%%%%%%%%%%%%%%%%%%%%%%%%%%%%

   \begin{frame} \frametitle{Randomized controlled experiments}

   \begin{block}
   {Another advantage}
   \begin{itemize}
   \item    As we'll see later, the other advantage of the randomized controlled experiment is that the only thing that the difference in rates between {\bf \color{red} treatment} and {\bf \color{blue} control} is randomness.


   \item    We will compute the chances of this later on. They will be small \dots
   \end{itemize}




   \end{block}
   \end{frame}

   %%%%%%%%%%%%%%%%%%%%%%%%%%%%%%%%%%%%%%%%%%%%%%%%%%%%%%%%%%%%

   \begin{frame} \frametitle{Observational studies}

   \begin{block}
   {Observational studies}
   \begin{itemize}
   \item Unlike in randomized controlled experiments, in {\em observational studies}, the subjects are assigned to {\bf \color{red} treatment} or {\bf \color{blue} control} by an {\em uncontrolled} mechanism.
   \item In a smoking / lung cancer study, subjects choose to smoke or not.
   \item Very often {\bf \color{red} treatment} or {\bf \color{blue} control} groups differ by more than just the treatment.
   \end{itemize}
   \end{block}
   \end{frame}

   %%%%%%%%%%%%%%%%%%%%%%%%%%%%%%%%%%%%%%%%%%%%%%%%%%%%%%%%%%%%

   \begin{frame} \frametitle{Observational studies}

   \begin{block}
   {Smoking \& socio-economic status}

   Smoking is related to socio-economic status, then
     \begin{itemize}
     \item     those in lower socio-economic status groups have less access to medical care
     \item will tend to have higher incidence of some diseases based on this fact alone
     \end{itemize}
   \end{block}

   \begin{block}
   {Association is not causation}
   \begin{itemize}
   \item    In children, shoe size is associated to reading ability.

   \item However, having big feet does not cause children to score high on reading tests.
   \end{itemize}
   \end{block}
   \end{frame}

   %%%%%%%%%%%%%%%%%%%%%%%%%%%%%%%%%%%%%%%%%%%%%%%%%%%%%%%%%%%%

   \begin{frame} \frametitle{Observational studies}

   \begin{block}
     {Confounding (excerpt from book)}
   {\em Confounding means there is a difference between the treatment and control groups -- other than treatment -- which affects the response being studied. A confounder is a third variable, associated with exposure and with disease.}
   \end{block}
   \end{frame}

   %%%%%%%%%%%%%%%%%%%%%%%%%%%%%%%%%%%%%%%%%%%%%%%%%%%%%%%%%%%%

   \begin{frame} \frametitle{Observational studies}

   \begin{block}
     {Confounding}
     \begin{itemize}
     \item Both reading and ability are associated to age.
     \item As children age, their feet grow.
     \item As children age, their reading improves.
     \end{itemize}
    \end{block}

    \begin{block}
    {The problem with confounding}

    \begin{itemize}
    \item  As it is impossible to rule out all possible confounding variables, establishing a causal link between, e.g. smoking and lung cancer can be difficult.

     \item Fisher, one of the greatest statisticians believed there was a confounding variable.

    \end{itemize}

   \end{block}
   \end{frame}

   %%%%%%%%%%%%%%%%%%%%%%%%%%%%%%%%%%%%%%%%%%%%%%%%%%%%%%%%%%%%

   \begin{frame} \frametitle{Simpson's paradox}

   \begin{block}
   {Simpson's paradox}
   Another difficulty with observational studies in which
   conclusions can be reversed \dots
   \end{block}
   \begin{block}
   {Sex bias in graduate admissions}
   {\small
   \begin{tabular}{c|rr|rr}
   Major & \# (Male) & \% (Male) & \# (Female) & \% (Female) \\ \hline
   \vdots & \vdots & \vdots & \vdots \\ \hline
   Total & 2691 & 44 & 1835 & 35
   \end{tabular}
   }
   \end{block}
   \end{frame}

   %%%%%%%%%%%%%%%%%%%%%%%%%%%%%%%%%%%%%%%%%%%%%%%%%%%%%%%%%%%%

   \begin{frame} \frametitle{Simpson's paradox}

   \begin{block}
   {Sex bias in graduate admissions}
   {\small
   \begin{tabular}{c|rr|rr}
   Major & \# (Male) & \% (Male) & \# (Female) & \% (Female) \\ \hline
   A & 825 & 62 & 108 & 82 \\
   B & 560 & 63 & 25 & 68 \\
   C & 325 & 37 & 593 & 34 \\
   D & 417 & 33 & 375  & 33 \\
   E & 191 & 28 & 393 & 24 \\
   F & 373 & 6 & 341 & 7 \\ \hline
   Total & 2691 & 44 & 1835 & 35
   \end{tabular}
   }
   \end{block}
   \end{frame}

   %%%%%%%%%%%%%%%%%%%%%%%%%%%%%%%%%%%%%%%%%%%%%%%%%%%%%%%%%%%%

   \begin{frame} \frametitle{Simpson's paradox}

   \begin{block}
   {Weighted average}
   {\small
   $$
   \begin{aligned}
   \text{Male} &= \frac{0.62 \times 933 + 0.63 \times 585 + 0.37 \times 918}{4526} \\
   & \qquad  +  \ \frac{0.33 \times 792 + 0.28 \times 584 + 0.06 \times 714}{4526} \\
   &= 39 \% \\
   \text{Female} &= \frac{0.82 \times 933 + 0.68 \times 585 + 0.34 \times 918}{4526} \\
   & \qquad + \  \frac{0.35 \times 792 + 0.24 \times 584 + 0.07 \times 714}{4526} \\
   &= 43 \%
   \end{aligned}
   $$
   }

   The women were applying to harder majors \dots

   \end{block}
   \end{frame}

   %%%%%%%%%%%%%%%%%%%%%%%%%%%%%%%%%%%%%%%%%%%%%%%%%%%%%%%%%%%%

   \begin{frame} 

   \end{frame}

   \end{document}
