   \documentclass[handout]{beamer}



   \mode<presentation>
   {
     \usetheme{PaloAlto}
   \setbeamertemplate{footline}[page number]

     \setbeamercolor{sidebar}{bg=white, fg=black}
     \setbeamercolor{frametitle}{bg=white, fg=black}
     % or ...
     \setbeamercolor{logo}{bg=white}
     \setbeamercolor{block body}{parent=normal text,bg=white}
     \setbeamercolor{author in sidebar}{fg=black}
     \setbeamercolor{title in sidebar}{fg=black}


     \setbeamercolor*{block title}{use=structure,fg=structure.fg,bg=structure.fg!20!bg}
     \setbeamercolor*{block title alerted}{use=alerted text,fg=alerted text.fg,bg=alerted text.fg!20!bg}
     \setbeamercolor*{block title example}{use=example text,fg=example text.fg,bg=example text.fg!20!bg}


     \setbeamercolor{block body}{parent=normal text,use=block title,bg=block title.bg!50!bg}
     \setbeamercolor{block body alerted}{parent=normal text,use=block title alerted,bg=block title alerted.bg!50!bg}
     \setbeamercolor{block body example}{parent=normal text,use=block title example,bg=block title example.bg!50!bg}

     % or ...

     \setbeamercovered{transparent}
     % or whatever (possibly just delete it)
     \logo{\resizebox{!}{1.5cm}{\href{\basename{R}}{\includegraphics{image}}}}
   }

   \mode<handout>
   {
     \usetheme{PaloAlto}
     \usecolortheme{default}
     \setbeamercolor{sidebar}{bg=white, fg=black}
     \setbeamercolor{frametitle}{bg=white, fg=black}
     % or ...
     \setbeamercolor{logo}{bg=white}
     \setbeamercolor{block body}{parent=normal text,bg=white}
     \setbeamercolor{author in sidebar}{fg=black}
     \setbeamercolor{title in sidebar}{fg=black}
     \setbeamercovered{transparent}
     % or whatever (possibly just delete it)
     \logo{}
   }

   \usepackage{epsdice}
   \usepackage[latin1]{inputenc}
   \usepackage{graphics}
   \usepackage{amsmath,eepic,epic}

   \newcommand{\figslide}[3]{
   \begin{frame}
   \frametitle{#1}
     \begin{center}
     \resizebox{!}{2.7in}{\includegraphics{#2}}    
     \end{center}
   {#3}
   \end{frame}
   }

   \newcommand{\fighslide}[4]{
   \begin{frame}
   \frametitle{#1}
     \begin{center}
     \resizebox{!}{#4}{\includegraphics{#2}}    
     \end{center}
   {#3}
   \end{frame}
   }

   \newcommand{\figwref}[1]{
   \href{#1}{\tiny \tt #1}}

   \newcommand{\B}[1]{\beta_{#1}}
   \newcommand{\Bh}[1]{\widehat{\beta}_{#1}}
   \newcommand{\V}{\text{Var}}
   \newcommand{\Cov}{\text{Cov}}
   \newcommand{\Vh}{\widehat{\V}}
   \newcommand{\s}{\sigma}
   \newcommand{\sh}{\widehat{\sigma}}

   \newcommand{\argmax}[1]{\mathop{\text{argmax}}_{#1}}
   \newcommand{\argmin}[1]{\mathop{\text{argmin}}_{#1}}
   \newcommand{\Ee}{\mathbb{E}}
   \newcommand{\Pp}{\mathbb{P}}
   \newcommand{\real}{\mathbb{R}}
   \newcommand{\Ybar}{\overline{Y}}
   \newcommand{\Yh}{\widehat{Y}}
   \newcommand{\Xbar}{\overline{X}}
   \newcommand{\Tr}{\text{Tr}}


   \newcommand{\model}{{\cal M}}

   \newcommand{\figvskip}{-0.7in}
   \newcommand{\fighskip}{-0.3in}
   \newcommand{\figheight}{3.5in}

   \newcommand{\Rcode}[1]{{\bf \tt #1 }}
   \newcommand{\Rtcode}[1]{{\tiny \bf \tt #1 }}
   \newcommand{\Rscode}[1]{{\small \bf \tt #1 }}

   \newcommand{\RR}{{\tt R} \;}
   \newcommand{\basename}[1]{http://stats60.stanford.edu/#1}
   \newcommand{\dataname}[1]{\basename{data/#1}}
   \newcommand{\Rname}[1]{\basename{R/#1}}

   \newcommand{\mycolor}[1]{{\color{blue} #1}}
   \newcommand{\basehref}[2]{\href{\basename{#1}}{\mycolor{#2}}}
   \newcommand{\Rhref}[2]{\href{\basename{R/#1}}{\mycolor{#2}}}
   \newcommand{\datahref}[2]{\href{\dataname{#1}}{\mycolor{#2}}}
   \newcommand{\X}{\pmb{X}}
   \newcommand{\Y}{\pmb{Y}}
   \newcommand{\be}{\pmb{varepsilon}}
   \newcommand{\logit}{\text{logit}}


   \title{Statistics 60: Introduction to Statistical Methods}
   \subtitle{Chapter 26: Tests of Significance} 
   \author{}% {Jonathan Taylor \\
   %Department of Statistics \\
   %Stanford University
   %}


   \begin{document}

   \begin{frame}
   \titlepage
   \end{frame}

   %%%%%%%%%%%%%%%%%%%%%%%%%%%%%%%%%%%%%%%%%%%%%%%%%%%%%%%%%%%%

   \begin{frame} \frametitle{Testing hypotheses}

   \begin{block}
   {Online gaming}
   \begin{itemize}
   \item If you play online poker or roulette, how do you know it's fair?

   \item I placed 10 bets on {\color{red} RED} on
   \href{http://www.roulette4fun.com/roulette-games/classic-roulette}{http://www.roulette4fun.com/roulette-games/classic-roulette}.

   \item These are the results: [0,0,1,0,1,0,0,1,0,1].

   \item Is the game fair?

   \end{itemize}
   \end{block}
   \end{frame}

   %%%%%%%%%%%%%%%%%%%%%%%%%%%%%%%%%%%%%%%%%%%%%%%%%%%%%%%%%%%%

   \begin{frame} \frametitle{Testing hypotheses}

   \begin{block}
   {Example (continued)}
   \begin{itemize}
   \item I {\em observed} 4 successes.
   \item If the roulette wheel is fair,
   I would expect to see $10 \times 18/ 38=4.7$ successes, give or take
   $\sqrt{10} \times \sqrt{18/38 \times 20/38}=1.4$.

   \item In standardized units, my observed 4 successes converts to
   $$
   \frac{{\color{orange} 4} - {\color{blue} 4.7}}{{\color{blue} 1.4}} = {\color{orange} -0.5}
   $$

   \item This seems reasonable, it is not very large relative to 1 and it could
   be just a chance error. After all, chance errors are typically size 1 in standardized units.


   \end{itemize}
   \end{block}
   \end{frame}

   %%%%%%%%%%%%%%%%%%%%%%%%%%%%%%%%%%%%%%%%%%%%%%%%%%%%%%%%%%%%

   \begin{frame} \frametitle{Testing hypotheses}

   \begin{block}
   {Example (continued)}
   \begin{itemize}
   \item I placed 10 more identical bets, with results
   [1,0,1,1,1,1,0,1,0,0]. So, in 20 bets, I have observed 10 successes.
   \item If the roulette game is fair, I would expect
   to see $20 \times 18/ 38=9.5$  successes, give or take
   $\sqrt{20} \times \sqrt{18/38 \times 20/38}=2.2$.

   \item In standardized units, my observed 10 successes converts to
   $$
   \frac{{\color{orange} 10} - {\color{blue} 9.5}}{{\color{blue} 2.3}} = {\color{orange} 0.2}
   $$

   \item Again, this seems reasonable, it is not very large relative to 1 and it could
   be just a chance error. After all, chance errors are typically size 1 in standardized units.


   \end{itemize}
   \end{block}
   \end{frame}

   %%%%%%%%%%%%%%%%%%%%%%%%%%%%%%%%%%%%%%%%%%%%%%%%%%%%%%%%%%%%

   \begin{frame} \frametitle{Testing hypotheses}

   \begin{block}
   {What can I conclude?}
   \begin{itemize}
   \item Based on betting on {\color{red} RED}, I cannot conclude that the roulette
   wheel is rigged.

   \item It is possible that some other bets might not have matched the expected number of
   successes as well. I can only draw a conclusion based on my 20 bets on {\color{red} RED}.

   \item In statistical terms, the statement ``the roulette game is rigged'' is called
   a {\em hypothesis}. In this case, we call it the {\em alternative hypothesis}.

   \item Alternative to what? It is an alternative to the {\em null hypothesis} which is
   ``the roulette game is fair.''


   \end{itemize}
   \end{block}
   \end{frame}

   %%%%%%%%%%%%%%%%%%%%%%%%%%%%%%%%%%%%%%%%%%%%%%%%%%%%%%%%%%%%

   \begin{frame} \frametitle{Testing hypotheses}

   \begin{block}
   {Null and alternative hypotheses}
   \begin{itemize}

   \item The naming of the hypotheses corresponds to an ``innocent until proven guilty'' approach.

   \item Since our observations (in standardized units) seem attributable to chance
   variation, we decided we cannot declare the null hypothesis to be false. Or,
   we cannot reject the null hypothesis.

   \item In
   legalese, ``there is reasonable doubt to the guilt of the roulette game so we do not convict''.

   \end{itemize}
   \end{block}
   \end{frame}

   %%%%%%%%%%%%%%%%%%%%%%%%%%%%%%%%%%%%%%%%%%%%%%%%%%%%%%%%%%%%

   \begin{frame} \frametitle{Testing hypotheses}

   \begin{block}
   {A different scenario}
   \begin{itemize}

   \item What if the second 10 results were [0,0,0,0,0,0,0,0,0,0]? Then,
   we would have observed only 4 successes in 20 draws from our 0-1 box.

   \item In standardized units, this 4 converts to
   $$
   \frac{{\color{orange} 4} - {\color{blue} 9.5}}{{\color{blue} 2.3}} = {\color{orange} -2.5}
   $$

   \item This would be a little suspect \dots

   \item In fact, if the roulette wheel were fair, there is only a 1.2\% chance
   that I would have had so few successes in 20 bets.

   \end{itemize}
   \end{block}
   \end{frame}

   %CODE
       % from matplotlib import rc
   % import pylab, numpy as np, sys
   % np.random.seed(0);import random; random.seed(0)
   % import pylab, numpy as np
   % x = np.linspace(-4,4,101)
   % y = np.exp(-x**2/2) / np.sqrt(2*np.pi)
   % x2 = np.linspace(-4,-2.25,101)
   % y2 = np.exp(-x2**2/2) / np.sqrt(2*np.pi)
   % pylab.plot(x,y*100, linewidth=2)
   % xf, yf = pylab.poly_between(x2, 0*x2, y2*100)
   % pylab.fill(xf, yf, facecolor='red', hatch='\\', alpha=0.5)
   % pylab.gca().set_xlabel('standardized units')
   % pylab.gca().set_ylabel('% per standardized unit')
   % #pylab.gca().set_xlim([-2,4])
   % #pylab.gca().set_yticks([])
   % 


   \begin{frame}
   \frametitle{What are the chances?}
   \begin{center}
   \resizebox{!}{2.7in}{\includegraphics{./images/inline/6e16e95105.pdf}}    
   \end{center}
   Continuity correction yields $z=-2.25$, area is 1.2\%.
   \end{frame}

   %%%%%%%%%%%%%%%%%%%%%%%%%%%%%%%%%%%%%%%%%%%%%%%%%%%%%%%%%%%%

   \begin{frame} \frametitle{Testing hypotheses}

   \begin{block}
   {$P$-values}
   \begin{itemize}

   \item The chances we computed are the chances, if the roulette game was fair, that we would observe a standardized less than our observed standardized value of {\color{orange} -2.5}.

   \item In general, if we test a null hypothesis with
   some {\color{orange} observed data} or {\color{orange} observed test statistic}, the {\color{orange} $P$-value} is
   the chance, assuming the null hypothesis is true, that
   we would observe such an extreme test statistic.

   \item In this test, we call the {\color{orange} observed test statistic}
   a {\color{orange} $z$-statistic} or {\color{orange} $z$-score}.
   \item These tests are called {\em \color{blue} $z$ tests}.
   \item Why color the {\color{orange} $P$-value} orange? Because
   it is random \dots

   \end{itemize}
   \end{block}
   \end{frame}

   %%%%%%%%%%%%%%%%%%%%%%%%%%%%%%%%%%%%%%%%%%%%%%%%%%%%%%%%%%%%

   \begin{frame} \frametitle{Testing hypotheses}

   \begin{block}
   {$P$-values}
   \begin{itemize}

   \item {\bf \color{red} The P-value is NOT the chances
   that the null hypothesis is correct!}

   \item Why not?
   \begin{itemize}
   \item The online roulette game is fair or it is not. The null
   hypothesis is true or it is false.
   \item If the {\color{orange} $P$-value} were the chances
    the null hypothesis is correct, these chances would be {\color{orange} random} \dots
   \end{itemize}

   \item There is a methodology, called Bayesian statistics in which
   one {\em can} compute the chances the null hypothesis is correct \dots

   \end{itemize}
   \end{block}
   \end{frame}

   %%%%%%%%%%%%%%%%%%%%%%%%%%%%%%%%%%%%%%%%%%%%%%%%%%%%%%%%%%%%

   \begin{frame} \frametitle{Testing hypotheses}

   \begin{block}
   {Chances the roulette wheel is fair}
   \begin{itemize}

   \item Suppose we declare, before seeing data: ``the probability
   the success rate on {\color{red} RED} is 18/38 is 70\%, and the probability
   the success rate on {\color{red} RED} is 12/38 is 30\%''.

   \item Call these two hypotheses $H_0, H_1$ and we have just said
   $P({\color{orange} \text{$H_0$ is true}})=0.7, P({\color{orange} \text{$H_1$ is true}})=0.3$.

   \end{itemize}
   \end{block}
   \end{frame}

   %%%%%%%%%%%%%%%%%%%%%%%%%%%%%%%%%%%%%%%%%%%%%%%%%%%%%%%%%%%%

   \begin{frame} \frametitle{Testing hypotheses}

   \begin{block}
   {Chances the roulette wheel is fair}
   \begin{itemize}

   \item Suppose now, we observe 4 success from 20 bets but we
   do not know whether they were from the fair game, $H_0$ or
   the unfair game $H_1$.

   \item Bayes' rule says
   $$
   \begin{aligned}
   \lefteqn{P(\text{$H_0$ is true} | \text{4 out of 20 successes})} \\
    &  \qquad =
   \frac{.7 \times \binom{20}{4} (\frac{18}{38})^4 (\frac{20}{38})^{16}}{.7 \times \binom{20}{4} (\frac{18}{38})^4 (\frac{20}{38})^{16} + .3 \times \binom{20}{4} (\frac{12}{38})^4 (\frac{26}{38})^{16}} \\
   & \qquad = 15\%.
   \end{aligned}
   $$
   \item Are the chances above {\color{blue} not random} or {\color{orange} random}?
   \end{itemize}
   \end{block}
   \end{frame}

   %%%%%%%%%%%%%%%%%%%%%%%%%%%%%%%%%%%%%%%%%%%%%%%%%%%%%%%%%%%%

   \begin{frame} \frametitle{Testing hypotheses}

   \begin{block}
   {A second testing scenario}
   \begin{itemize}

   \item Suppose now we want to see the efficacy of a
   new drug on blood pressure.

   \item Our study design is: we will treat
   a large patient population with the drug and measure their
   blood pressure before and after taking the drug.

   \item One way to conclude that the drug is effective if the blood pressure has decreased. That is,
   if the average difference is negative.

   \end{itemize}
   \end{block}
   \end{frame}

   %%%%%%%%%%%%%%%%%%%%%%%%%%%%%%%%%%%%%%%%%%%%%%%%%%%%%%%%%%%%

   \begin{frame} \frametitle{Testing hypotheses}

   \begin{block}
   {Setting up the test}
   \begin{itemize}

   \item We could set this up as drawing from a box of {\em differences
   in blood pressure}.

   \item The {\em null hypothesis}, $H_0$ is: ``the average difference is zero.''

   \item The {\em alternative hypothesis}, $H_a$, is: ``the average difference is less than zero.''

   \item Sometimes, people will test the alternative, $H_a$: ``the
   average difference is not zero.''

   \item We test the null with observed data by estimating
    the average difference and converting to standardized units.
   \end{itemize}
   \end{block}
   \end{frame}

   %CODE
       % from matplotlib import rc
   % import pylab, numpy as np, sys
   % np.random.seed(0);import random; random.seed(0)
   % import random
   % X = np.mgrid[0:1:10j,0:1:5j].reshape((2,50)) + np.random.sample((2,50)) * 0.05
   % X = X.T
   % sample = [random.randint(-15,0) for _  in range(50)]
   % for i in range(50):
   %     pylab.text(X[i,0], X[i,1], '%d' % sample[i])
   % 
   %     pylab.gca().set_xticks([]);    pylab.gca().set_xlim([-0.1,1.1])
   %     pylab.gca().set_yticks([]);    pylab.gca().set_ylim([-0.1,1.1])
   % pylab.title("average(sample)=-7.4, SD(sample)=4.8") # % (np.mean(sample), np.std(sample)))
   % 


   \begin{frame}
   \frametitle{Sample of blood pressures}
   \begin{center}
   \resizebox{!}{2.7in}{\includegraphics{./images/inline/3e7e808490.pdf}}    
   \end{center}
   Sample of 50
   \end{frame}

   %%%%%%%%%%%%%%%%%%%%%%%%%%%%%%%%%%%%%%%%%%%%%%%%%%%%%%%%%%%%

   \begin{frame} \frametitle{Testing hypotheses}

   \begin{block}
   {Evaluating  the test}
   \begin{itemize}

   \item Our observed average is ${-\color{orange} 7.4}$. We estimate its SE
   to be ${\color{orange} 4.8 / \sqrt{50}=0.7}$.

    \item In standardized units, our observed average converts to
    $$
    \frac{{\color{orange}-7.4} - {\color{blue} 0}}{\color{orange} 0.7} \approx - 10
    $$

   \item The {\color{orange} $P$-value} is 0: there is virtually no chance
   a standard normal would ever be so small. We reject the null hypothesis $H_0$
   and conclude $H_a$: ``the average difference is negative.''


   \end{itemize}
   \end{block}
   \end{frame}

   %%%%%%%%%%%%%%%%%%%%%%%%%%%%%%%%%%%%%%%%%%%%%%%%%%%%%%%%%%%%

   \begin{frame} \frametitle{Testing hypotheses}

   \begin{block}
   {Example (continued)}
   \begin{itemize}

   \item Suppose that our null hypothesis was different. We might
   have begun with the null hypothesis $H_0$: ``the average decrease in
   blood pressure will be 7 mm Hg'' with alternative $H_a$: ``the
   average decrease in blood pressure is not 7mm Hg''.

    \item How do we test this hypothesis?

    \item Well, under this null hypothesis our observed average converts to
    $$
    \frac{{\color{orange}-7.4} - ({\color{blue} -7})}{\color{orange} 0.7} \approx -0.6
    $$

    \item This test is {\em two-sided}: we did not say greater
    than or larger than \dots


   \end{itemize}
   \end{block}
   \end{frame}

   %CODE
       % from matplotlib import rc
   % import pylab, numpy as np, sys
   % np.random.seed(0);import random; random.seed(0)
   % import pylab, numpy as np
   % x = np.linspace(-4,4,101)
   % y = np.exp(-x**2/2) / np.sqrt(2*np.pi)
   % x2 = np.linspace(-4,-.6,101)
   % y2 = np.exp(-x2**2/2) / np.sqrt(2*np.pi)
   % pylab.plot(x,y*100, linewidth=2)
   % xf, yf = pylab.poly_between(x2, 0*x2, y2*100)
   % pylab.fill(xf, yf, facecolor='red', hatch='\\', alpha=0.5)
   % 
   % x2 = np.linspace(.6,4,101)
   % y2 = np.exp(-x2**2/2) / np.sqrt(2*np.pi)
   % pylab.plot(x,y*100, linewidth=2)
   % xf, yf = pylab.poly_between(x2, 0*x2, y2*100)
   % pylab.fill(xf, yf, facecolor='red', hatch='\\', alpha=0.5)
   % 
   % pylab.gca().set_xlabel('standardized units')
   % pylab.gca().set_ylabel('% per standardized unit')
   % #pylab.gca().set_xlim([-2,4])
   % #pylab.gca().set_yticks([])
   % 


   \begin{frame}
   \frametitle{What are the chances?}
   \begin{center}
   \resizebox{!}{2.7in}{\includegraphics{./images/inline/9ed4b03716.pdf}}    
   \end{center}
   Area is 55\%, we cannot reject $H_0: \text{average(difference)}=-7$.
   \end{frame}

   %%%%%%%%%%%%%%%%%%%%%%%%%%%%%%%%%%%%%%%%%%%%%%%%%%%%%%%%%%%%

   \begin{frame} \frametitle{Testing hypotheses}

   \begin{block}
   {How small should the chances be?}
   \begin{itemize}

   \item In our examples so far, it has been fairly clear
   which of the null or alternative is more believable.

   \item In practice, we must decide a threshold at which to
   reject $H_0$.

   \item A common choice is to use a threshold of 5\%. We call this
   threshold the {\em level} or {\em size} of the test.

   \item The book declares a {\em \color{orange} $P$-value} of $5\%$ or less to be ``significant'',
   $1\%$ or less to be ``highly significant.''
   \end{itemize}
   \end{block}
   \end{frame}

   %%%%%%%%%%%%%%%%%%%%%%%%%%%%%%%%%%%%%%%%%%%%%%%%%%%%%%%%%%%%

   \begin{frame} \frametitle{Testing hypotheses}

   \begin{block}
   {Rejection rule}
   \begin{itemize}

   \item Knowing the null and alternative hypotheses and the size of the test,
   we can define a {\em \color{blue} rejection rule}.

   \item For example, if the size is 5\%, and
   $$
   \begin{aligned}
     H_0 &= \text{average difference is 0 mm Hg} \\
     H_a &= \text{average difference is negative} \\
   \end{aligned}
   $$
   \item Then, we reject $H_0$ if our $z$ statistic is
   less than {\color{blue} -1.65}.

   \end{itemize}
   \end{block}
   \end{frame}

   %CODE
       % from matplotlib import rc
   % import pylab, numpy as np, sys
   % np.random.seed(0);import random; random.seed(0)
   % import pylab, numpy as np
   % x = np.linspace(-4,4,101)
   % y = np.exp(-x**2/2) / np.sqrt(2*np.pi)
   % x2 = np.linspace(-4,-1.65,101)
   % y2 = np.exp(-x2**2/2) / np.sqrt(2*np.pi)
   % pylab.plot(x,y*100, linewidth=2)
   % xf, yf = pylab.poly_between(x2, 0*x2, y2*100)
   % pylab.fill(xf, yf, facecolor='red', hatch='\\', alpha=0.5)
   % pylab.gca().set_xlabel('standardized units')
   % pylab.gca().set_ylabel('% per standardized unit')
   % #pylab.gca().set_xlim([-2,4])
   % #pylab.gca().set_yticks([])
   % 


   \begin{frame}
   \frametitle{One sided test}
   \begin{center}
   \resizebox{!}{2.7in}{\includegraphics{./images/inline/73222db9d8.pdf}}    
   \end{center}
   {\color{blue} 5\% rejection rule} when alternative is negative \dots
   \end{frame}

   %%%%%%%%%%%%%%%%%%%%%%%%%%%%%%%%%%%%%%%%%%%%%%%%%%%%%%%%%%%%

   \begin{frame} \frametitle{Testing hypotheses}

   \begin{block}
   {Example (continued)}
   \begin{itemize}

   \item Suppose the alternative is now $H_a$: ``the average difference is not 0mm Hg''.

   \item Do we reject $H_0$:``the average difference is 0mm Hg''?

    \item In standardized units, our observed average converts to
    $$
    \frac{{\color{orange}-7.4} - {\color{blue} 0}}{\color{orange} 0.7} \approx -10
    $$

   \item Now, -10 is extremely unlikely under $H_0$ but it is likely
   for values some values of the average difference under $H_a$.

   \item We reject this $H_0$ when the $z$-score is large in absolute value.
   \end{itemize}
   \end{block}
   \end{frame}

   %CODE
       % from matplotlib import rc
   % import pylab, numpy as np, sys
   % np.random.seed(0);import random; random.seed(0)
   % import pylab, numpy as np
   % x = np.linspace(-4,4,101)
   % y = np.exp(-x**2/2) / np.sqrt(2*np.pi)
   % 
   % x2 = np.linspace(2,4,101)
   % y2 = np.exp(-x2**2/2) / np.sqrt(2*np.pi)
   % xf, yf = pylab.poly_between(x2, 0*x2, y2*100)
   % pylab.fill(xf, yf, facecolor='red', hatch='\\', alpha=0.5)
   % 
   % 
   % x2 = np.linspace(-4,-2,101)
   % y2 = np.exp(-x2**2/2) / np.sqrt(2*np.pi)
   % xf, yf = pylab.poly_between(x2, 0*x2, y2*100)
   % pylab.fill(xf, yf, facecolor='red', hatch='\\', alpha=0.5)
   % 
   % pylab.plot(x,y*100, linewidth=2)
   % pylab.gca().set_xlabel('standardized units')
   % pylab.gca().set_ylabel('% per standardized unit')
   % #pylab.gca().set_xlim([-2,4])
   % #pylab.gca().set_yticks([])
   % 


   \begin{frame}
   \frametitle{Two sided test}
   \begin{center}
   \resizebox{!}{2.7in}{\includegraphics{./images/inline/bbd6d849ef.pdf}}    
   \end{center}
   {\color{blue} 5\% rejection rule} when alternative is positive or negative \dots
   \end{frame}

   %%%%%%%%%%%%%%%%%%%%%%%%%%%%%%%%%%%%%%%%%%%%%%%%%%%%%%%%%%%%

   \begin{frame} \frametitle{Testing hypotheses}

   \begin{block}
   {Example (continued)}
   \begin{itemize}

   \item Suppose the alternative is now $H_a$: ``the average difference is positive''.
   \item Do we reject $H_0$:``the average difference is 0mm Hg''?

    \item In standardized units, our observed average converts to
    $$
    \frac{{\color{orange}-7.4} - {\color{blue} 0}}{\color{orange} 0.7} \approx -10
    $$

   \item While -10 is extremely unlikely under $H_0$ it is even more
   unlikely under $H_a$.

   \item It seems reasonable to conclude that neither $H_0$ nor
   $H_a$ is true.

   \item But, to conclude $H_a$ is true, we should only reject this $H_0$ when the $z$-score is positive \dots
   \end{itemize}
   \end{block}
   \end{frame}

   %CODE
       % from matplotlib import rc
   % import pylab, numpy as np, sys
   % np.random.seed(0);import random; random.seed(0)
   % import pylab, numpy as np
   % x = np.linspace(-4,4,101)
   % y = np.exp(-x**2/2) / np.sqrt(2*np.pi)
   % x2 = np.linspace(1.65,4,101)
   % y2 = np.exp(-x2**2/2) / np.sqrt(2*np.pi)
   % pylab.plot(x,y*100, linewidth=2)
   % xf, yf = pylab.poly_between(x2, 0*x2, y2*100)
   % pylab.fill(xf, yf, facecolor='red', hatch='\\', alpha=0.5)
   % pylab.gca().set_xlabel('standardized units')
   % pylab.gca().set_ylabel('% per standardized unit')
   % #pylab.gca().set_xlim([-2,4])
   % #pylab.gca().set_yticks([])
   % 


   \begin{frame}
   \frametitle{One sided test}
   \begin{center}
   \resizebox{!}{2.7in}{\includegraphics{./images/inline/87f9fdb9ee.pdf}}    
   \end{center}
   {\color{blue} 5\% rejection rule} when alternative is positive \dots
   \end{frame}

   %%%%%%%%%%%%%%%%%%%%%%%%%%%%%%%%%%%%%%%%%%%%%%%%%%%%%%%%%%%%

   \begin{frame} \frametitle{Testing hypotheses}

   \begin{block}
   {One-sided vs. two-sided}
   \begin{itemize}

   \item If we want to conclude a one-sided alternative like
     $H_a$:``the average difference in blood pressure is less than -7 mm Hg''.
   \item Then, we can take the null hypothesis to be
   $H_0$:``the average difference in blood pressure is greater than or equal to -7 mm Hg''.
   We reject for $z$-scores that are negative and large
   in absolute value.

   \item On the other hand, if we want to conclude a two-sided
   alternative like $H_a$:``the average difference in blood pressure is
    not -7 mm Hg''.

   \item Then, we can take the null hypothesis to be
   $H_0$:``the average difference
   in blood pressure is equal to -7 mm Hg''. We reject for large $z$-scores
   in absolute value.


   \end{itemize}
   \end{block}
   \end{frame}

   %%%%%%%%%%%%%%%%%%%%%%%%%%%%%%%%%%%%%%%%%%%%%%%%%%%%%%%%%%%%

   \begin{frame} \frametitle{Normal approximation}

    \begin{block}
     {Normal approximation and hypothesis tests}
     \begin{itemize}
     \item If a normal approximation holds
       for ${\color{orange} \widehat{\theta}}$
       then we can test the null hypothesis $H_0: {\color{blue} \theta=\theta_0}$ against $H_a: {\color{blue} \theta \neq \theta_0}$.
     \item For instance, our first null hypothesis was $\theta_0=0$. In the
     second, $\theta_0=-7$.

     \item The test statistic, called a {\color{orange} $z$ score}
     for testing $H_0:{\color{blue} \theta=\theta_0}$
     is
      $$
      {\color{orange} z} = \frac{{\color{orange} \widehat{\theta}} - {\color{blue} \theta_0}}{{\color{blue}\text{SE}({\color{orange} \widehat{\theta}})}}  \approx \frac{{\color{orange} \widehat{\theta}} - {\color{blue} \theta_0}}{{\color{orange}\widehat{\text{SE}({ \widehat{\theta}})}}}
      $$

     \item We call ${\color{orange} z}$ a $Z$-statistic or a $Z$-score.
     \item If $H_0$ is true, then ${\color{orange} z}$ follows the standard normal curve.

     \item If $H_0$ is not true, then ${\color{orange} z}$ does not usually follow the standard normal curve. If it does, you have a very poor test \dots
     \end{itemize}
   \end{block}
   \end{frame}

   %%%%%%%%%%%%%%%%%%%%%%%%%%%%%%%%%%%%%%%%%%%%%%%%%%%%%%%%%%%%

   \begin{frame} \frametitle{Normal approximation}

    \begin{block}
     {Normal approximation and hypothesis tests}
     \begin{itemize}
     \item If $H_0$ is not true, then ${\color{orange} z}$ does not usually follow the standard normal curve. If it does, you have a very poor test \dots

     \item It may follow a normal curve with mean $\neq 0$.

     \item The logic of the hypothesis test is as follows: if $H_0$ is true, then
     our observed test statistic should be a ``typical value'' under $H_0$.

     \item The {\color{orange} $P$-value} depends on what $H_a$ is.

     \item It is often easier to use the rejection rule instead
     of the $P$-value.

     \item For null hypotheses like $H_0:{\color{blue} \theta \leq \theta_0}$ and
     $H_0:{\color{blue} \theta \geq \theta_0}$ we use the rejection rules with
     the {\em \color{orange} same  $z$-score} \dots

     \end{itemize}
   \end{block}
   \end{frame}

   %CODE
       % from matplotlib import rc
   % import pylab, numpy as np, sys
   % np.random.seed(0);import random; random.seed(0)
   % import pylab, numpy as np
   % x = np.linspace(-4,4,101)
   % y = np.exp(-x**2/2) / np.sqrt(2*np.pi)
   % x2 = np.linspace(-4,-1.65,101)
   % y2 = np.exp(-x2**2/2) / np.sqrt(2*np.pi)
   % pylab.plot(x,y*100, linewidth=2)
   % xf, yf = pylab.poly_between(x2, 0*x2, y2*100)
   % pylab.fill(xf, yf, facecolor='red', hatch='\\', alpha=0.5)
   % pylab.gca().set_xlabel('standardized units')
   % pylab.gca().set_ylabel('% per standardized unit')
   % #pylab.gca().set_xlim([-2,4])
   % #pylab.gca().set_yticks([])
   % 


   \begin{frame}
   \frametitle{One sided test}
   \begin{center}
   \resizebox{!}{2.7in}{\includegraphics{./images/inline/73222db9d8.pdf}}    
   \end{center}
   {\color{blue} 5\% rejection rule} for $H_a: \theta < \theta_0, H_0:\theta \geq \theta_0$
   \end{frame}

   %CODE
       % from matplotlib import rc
   % import pylab, numpy as np, sys
   % np.random.seed(0);import random; random.seed(0)
   % import pylab, numpy as np
   % x = np.linspace(-4,4,101)
   % y = np.exp(-x**2/2) / np.sqrt(2*np.pi)
   % x2 = np.linspace(1.65,4,101)
   % y2 = np.exp(-x2**2/2) / np.sqrt(2*np.pi)
   % pylab.plot(x,y*100, linewidth=2)
   % xf, yf = pylab.poly_between(x2, 0*x2, y2*100)
   % pylab.fill(xf, yf, facecolor='red', hatch='\\', alpha=0.5)
   % pylab.gca().set_xlabel('standardized units')
   % pylab.gca().set_ylabel('% per standardized unit')
   % #pylab.gca().set_xlim([-2,4])
   % #pylab.gca().set_yticks([])
   % 


   \begin{frame}
   \frametitle{One sided test}
   \begin{center}
   \resizebox{!}{2.7in}{\includegraphics{./images/inline/87f9fdb9ee.pdf}}    
   \end{center}
   {\color{blue} 5\% rejection rule} for $H_a: \theta > \theta_0, H_0:\theta \leq \theta_0$
   \end{frame}

   %CODE
       % from matplotlib import rc
   % import pylab, numpy as np, sys
   % np.random.seed(0);import random; random.seed(0)
   % import pylab, numpy as np
   % x = np.linspace(-4,4,101)
   % y = np.exp(-x**2/2) / np.sqrt(2*np.pi)
   % 
   % x2 = np.linspace(2,4,101)
   % y2 = np.exp(-x2**2/2) / np.sqrt(2*np.pi)
   % xf, yf = pylab.poly_between(x2, 0*x2, y2*100)
   % pylab.fill(xf, yf, facecolor='red', hatch='\\', alpha=0.5)
   % 
   % 
   % x2 = np.linspace(-4,-2,101)
   % y2 = np.exp(-x2**2/2) / np.sqrt(2*np.pi)
   % xf, yf = pylab.poly_between(x2, 0*x2, y2*100)
   % pylab.fill(xf, yf, facecolor='red', hatch='\\', alpha=0.5)
   % 
   % pylab.plot(x,y*100, linewidth=2)
   % pylab.gca().set_xlabel('standardized units')
   % pylab.gca().set_ylabel('% per standardized unit')
   % #pylab.gca().set_xlim([-2,4])
   % #pylab.gca().set_yticks([])
   % 


   \begin{frame}
   \frametitle{Two sided test}
   \begin{center}
   \resizebox{!}{2.7in}{\includegraphics{./images/inline/bbd6d849ef.pdf}}    
   \end{center}
   {\color{blue} 5\% rejection rule} for $H_a: \theta \neq \theta_0, H_0: \theta = \theta_0$
   \end{frame}

   %%%%%%%%%%%%%%%%%%%%%%%%%%%%%%%%%%%%%%%%%%%%%%%%%%%%%%%%%%%%

   \begin{frame} \frametitle{Testing hypotheses}

   \begin{block}
   {Relation between hypothesis tests and confidence intervals}
   \begin{itemize}

   \item Which values are reasonable?


   \item Well, -7.4 is certainly a reasonable value if the true average
   difference were -7.4 because our $z$ score would be 0.

   \item Hence, we would not reject $H_0$:``the average difference is -7.4'' if we observed a sample average of -7.4

    \item The set of all values $\theta$ we would not reject $H_0$: ``the average difference is $\theta$'' at level 5\% is essentially the standard 95\% confidence interval!

    \item Therefore, one can test $H_0:$``the average difference is 0'' by
    checking to see whether 0 is in the confidence interval.

   \end{itemize}
   \end{block}
   \end{frame}

   %%%%%%%%%%%%%%%%%%%%%%%%%%%%%%%%%%%%%%%%%%%%%%%%%%%%%%%%%%%%

   \begin{frame} \frametitle{Back to roulette}

   \begin{block}
   {Testing fairness via a confidence interval}
   \begin{itemize}

   \item A 95\% confidence for the true {\color{red} RED} success rate
   (fair or not)
   based on our first 10 bets is
   $$
   {\color{orange} 0.4 \pm \sqrt{0.4 \times 0.6} / {\sqrt{10}}} =
   {\color{orange} 0.4 \pm 0.15}
   $$
   (This assumes the online roulette game is doing independent trials,
    thought not necessarily fair trials)

   \item The success rate for {\color{red} RED} in the fair model is
   ${\color{blue} 18/38 \approx 0.47}$.

    \item 0.47 is within our 95\% confidence interval. Therefore, we would
    not reject $H_0$:``the roulette table is fair'' at level 5\%.

    \item {\bf Note:} In this example, we only have 10 trials --
    to apply this logic we should have enough trials so the normal
    approximation holds.
   \end{itemize}
   \end{block}
   \end{frame}

   %%%%%%%%%%%%%%%%%%%%%%%%%%%%%%%%%%%%%%%%%%%%%%%%%%%%%%%%%%%%

   \begin{frame} \frametitle{Student's $T$}

   \begin{block}
   {Tests and confidence intervals for small samples}
   \begin{itemize}

   \item Our tests (and confidence intervals) have so far
   relied on normal approximations  (i.e. we have used A-104 to compute
   all chances).

   \item If the sample size is small, the normal approximations may
   not be very good.

   \item If the sample size is small, we can sometimes get
   good confidence intervals using a $T$ statistic.

   \item The formula for the $T$ statistic is almost identical
   to the $z$ statistic, it is the {\em chances} that can be quite different.

   \end{itemize}
   \end{block}
   \end{frame}

   %%%%%%%%%%%%%%%%%%%%%%%%%%%%%%%%%%%%%%%%%%%%%%%%%%%%%%%%%%%%

   \begin{frame} \frametitle{Student's $T$}

   \begin{block}
   {Tests and confidence intervals for small samples}
   \begin{itemize}

   \item Suppose the Gauss model holds
     $$
     {\color{orange} \text{measurement}} = {\color{blue} \text{true value}} + {\color{orange} \text{chance error}}
     $$
   \item {\bf And, the histogram of the error box is not too different
   from a normal probability histogram or curve}

   \item Then, there are very good confidence intervals
   even for very small samples \dots

   \item If the histogram of the error box is exactly
   a normal probability histogram, then these tests and confidence
   intervals are {\em exact}.
   \end{itemize}
   \end{block}
   \end{frame}

   %%%%%%%%%%%%%%%%%%%%%%%%%%%%%%%%%%%%%%%%%%%%%%%%%%%%%%%%%%%%

   \begin{frame} \frametitle{Student's $T$}

   \begin{block}
     {Example}
     \begin{itemize}
     \item Suppose we observed only 5 blood pressure changes: [-4,-6,-8,-2,-1].
     \item The average is -4.2 mm Hg, and the SD of the list is 2.6 mm Hg.
     \item Our usual $z$ score to test $H_0$: average difference $\geq 0$
     against $H_a$: average difference $<0$
     $$
     {\color{orange} z = \frac{-4.2}{2.6 / \sqrt{5}} \approx -3.7}
     $$
     \item The $T$ statistic replaces the SD of the list with SD$^+$ of the list
     which is 2.9 mm Hg. The $T$ statistic is
     $$
     {\color{orange} T = \frac{-4.2}{2.9 / \sqrt{5}} \approx -3.3}
     $$
     \end{itemize}
   \end{block}
   \end{frame}

   %%%%%%%%%%%%%%%%%%%%%%%%%%%%%%%%%%%%%%%%%%%%%%%%%%%%%%%%%%%%

   \begin{frame} \frametitle{Student's $T$}

   \begin{block}
   {What's different about the $T$ statistic?}
   \begin{itemize}

     \item It uses ${\color{orange} \text{SD}^+}$ instead
     of ${\color{orange} \text{SD}}$.
     \item Why does it use ${\color{orange} \text{SD}^+}$?
       \item     Because for small samples,
     it is a better estimate of ${\color{blue} \text{SD(box)}}$
     than ${\color{orange} \text{SD}}$.

   \item Unfortunately, though, the $T$ statistic does not follow the
   normal curve.
   \end{itemize}
   \end{block}
   \end{frame}

   %%%%%%%%%%%%%%%%%%%%%%%%%%%%%%%%%%%%%%%%%%%%%%%%%%%%%%%%%%%%

   \begin{frame} \frametitle{Student's $T$}

   \begin{block}
   {Computing the chances for the $T$ test}
   \begin{itemize}

   \item It {\em almost} follows the normal curve. For large
     samples, it gets closer and closer.

   \item For each sample size, there is a different curve, or
   probability histogram.

   \item These curves are indexed by what we call {\em degrees of freedom}.

   \item In this example, the degrees of freedom are $\# \text{sample}-1$.
   \end{itemize}
   \end{block}
   \end{frame}

   %CODE
       % from matplotlib import rc
   % import pylab, numpy as np, sys
   % np.random.seed(0);import random; random.seed(0)
   % import pylab, numpy as np
   % import scipy.stats
   % 
   % df = 4
   % x = np.linspace(-4,4,101)
   % y = np.exp(-x**2/2) / np.sqrt(2*np.pi)
   % pylab.plot(x,y*100, linewidth=2, label='Normal')
   % pylab.plot(x,scipy.stats.t.pdf(x, df)*100, linewidth=2, label=r'$T$, df=4')
   % 
   % pylab.gca().set_xlabel('standardized units')
   % pylab.gca().set_ylabel('% per standardized unit')
   % pylab.legend()
   % #pylab.gca().set_xlim([-2,4])
   % #pylab.gca().set_yticks([])
   % 


   \begin{frame}
   \frametitle{Student's $T$}
   \begin{center}
   \resizebox{!}{2.7in}{\includegraphics{./images/inline/42591bcc28.pdf}}    
   \end{center}
   Comparison with normal curve, degrees of freedom = 4
   \end{frame}

   %CODE
       % from matplotlib import rc
   % import pylab, numpy as np, sys
   % np.random.seed(0);import random; random.seed(0)
   % import pylab, numpy as np
   % import scipy.stats
   % 
   % df = 4
   % x = np.linspace(-4,4,101)
   % y = np.exp(-x**2/2) / np.sqrt(2*np.pi)
   % pylab.plot(x,y*100, linewidth=2, label='Normal')
   % pylab.plot(x,scipy.stats.t.pdf(x, df)*100, linewidth=2, label=r'$T$, df=4')
   % 
   % x2 = np.linspace(2,4,101)
   % y2 = np.exp(-x2**2/2) / np.sqrt(2*np.pi)
   % xf, yf = pylab.poly_between(x2, 0*x2, y2*100)
   % pylab.fill(xf, yf, facecolor='red', hatch='\\', alpha=0.5)
   % 
   % 
   % x2 = np.linspace(-4,-2,101)
   % y2 = np.exp(-x2**2/2) / np.sqrt(2*np.pi)
   % xf, yf = pylab.poly_between(x2, 0*x2, y2*100)
   % pylab.fill(xf, yf, facecolor='red', hatch='\\', alpha=0.5)
   % 
   % # The t region
   % 
   % x2 = np.linspace(scipy.stats.t.isf(0.025,df),4, 101)
   % y2 = scipy.stats.t.pdf(x2, df)
   % xf, yf = pylab.poly_between(x2, 0*x2, y2*100)
   % pylab.fill(xf, yf, facecolor='gray', hatch='\\', alpha=0.5)
   % 
   % 
   % 
   % x2 = np.linspace(-4,-scipy.stats.t.isf(0.025,df), 101)
   % y2 = scipy.stats.t.pdf(x2, df)
   % xf, yf = pylab.poly_between(x2, 0*x2, y2*100)
   % pylab.fill(xf, yf, facecolor='gray', hatch='\\', alpha=0.5)
   % 
   % pylab.gca().set_xlabel('standardized units')
   % pylab.gca().set_ylabel('% per standardized unit')
   % pylab.legend()
   % #pylab.gca().set_xlim([-2,4])
   % #pylab.gca().set_yticks([])
   % 


   \begin{frame}
   \frametitle{Student's $T$}
   \begin{center}
   \resizebox{!}{2.7in}{\includegraphics{./images/inline/57599c5280.pdf}}    
   \end{center}
   Comparison of two-sided {\color{blue} 5\% rejection rule}, df=4
   \end{frame}

   %CODE
       % from matplotlib import rc
   % import pylab, numpy as np, sys
   % np.random.seed(0);import random; random.seed(0)
   % import pylab, numpy as np
   % import scipy.stats
   % 
   % df = 20
   % x = np.linspace(-4,4,101)
   % y = np.exp(-x**2/2) / np.sqrt(2*np.pi)
   % pylab.plot(x,y*100, linewidth=2, label='Normal')
   % pylab.plot(x,scipy.stats.t.pdf(x, df)*100, linewidth=2, label=r'$T$, df=20')
   % 
   % pylab.gca().set_xlabel('standardized units')
   % pylab.gca().set_ylabel('% per standardized unit')
   % pylab.legend()
   % #pylab.gca().set_xlim([-2,4])
   % #pylab.gca().set_yticks([])
   % 


   \begin{frame}
   \frametitle{Student's $T$}
   \begin{center}
   \resizebox{!}{2.7in}{\includegraphics{./images/inline/9bcfab42cd.pdf}}    
   \end{center}
   Comparison with normal curve, degrees of freedom = 20
   \end{frame}

   %CODE
       % from matplotlib import rc
   % import pylab, numpy as np, sys
   % np.random.seed(0);import random; random.seed(0)
   % import pylab, numpy as np
   % import scipy.stats
   % 
   % df = 20
   % x = np.linspace(-4,4,101)
   % y = np.exp(-x**2/2) / np.sqrt(2*np.pi)
   % pylab.plot(x,y*100, linewidth=2, label='Normal')
   % pylab.plot(x,scipy.stats.t.pdf(x, df)*100, linewidth=2, label=r'$T$, df=20')
   % 
   % x2 = np.linspace(2,4,101)
   % y2 = np.exp(-x2**2/2) / np.sqrt(2*np.pi)
   % xf, yf = pylab.poly_between(x2, 0*x2, y2*100)
   % pylab.fill(xf, yf, facecolor='red', hatch='\\', alpha=0.5)
   % 
   % 
   % x2 = np.linspace(-4,-2,101)
   % y2 = np.exp(-x2**2/2) / np.sqrt(2*np.pi)
   % xf, yf = pylab.poly_between(x2, 0*x2, y2*100)
   % pylab.fill(xf, yf, facecolor='red', hatch='\\', alpha=0.5)
   % 
   % # The t region
   % 
   % x2 = np.linspace(scipy.stats.t.isf(0.025,df),4, 101)
   % y2 = scipy.stats.t.pdf(x2, df)
   % xf, yf = pylab.poly_between(x2, 0*x2, y2*100)
   % pylab.fill(xf, yf, facecolor='gray', hatch='\\', alpha=0.5)
   % 
   % 
   % 
   % x2 = np.linspace(-4,-scipy.stats.t.isf(0.025,df), 101)
   % y2 = scipy.stats.t.pdf(x2, df)
   % xf, yf = pylab.poly_between(x2, 0*x2, y2*100)
   % pylab.fill(xf, yf, facecolor='gray', hatch='\\', alpha=0.5)
   % 
   % pylab.gca().set_xlabel('standardized units')
   % pylab.gca().set_ylabel('% per standardized unit')
   % pylab.legend()
   % #pylab.gca().set_xlim([-2,4])
   % #pylab.gca().set_yticks([])
   % 


   \begin{frame}
   \frametitle{Student's $T$}
   \begin{center}
   \resizebox{!}{2.7in}{\includegraphics{./images/inline/96c10c598f.pdf}}    
   \end{center}
   Comparison of two-sided {\color{blue} 5\% rejection rule}, df=20
   \end{frame}

   %%%%%%%%%%%%%%%%%%%%%%%%%%%%%%%%%%%%%%%%%%%%%%%%%%%%%%%%%%%%

   \begin{frame} 

   \end{frame}

   \end{document}
