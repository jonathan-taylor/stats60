   \documentclass[handout]{beamer}



   \mode<presentation>
   {
     \usetheme{PaloAlto}
   \setbeamertemplate{footline}[page number]

     \setbeamercolor{sidebar}{bg=white, fg=black}
     \setbeamercolor{frametitle}{bg=white, fg=black}
     % or ...
     \setbeamercolor{logo}{bg=white}
     \setbeamercolor{block body}{parent=normal text,bg=white}
     \setbeamercolor{author in sidebar}{fg=black}
     \setbeamercolor{title in sidebar}{fg=black}


     \setbeamercolor*{block title}{use=structure,fg=structure.fg,bg=structure.fg!20!bg}
     \setbeamercolor*{block title alerted}{use=alerted text,fg=alerted text.fg,bg=alerted text.fg!20!bg}
     \setbeamercolor*{block title example}{use=example text,fg=example text.fg,bg=example text.fg!20!bg}


     \setbeamercolor{block body}{parent=normal text,use=block title,bg=block title.bg!50!bg}
     \setbeamercolor{block body alerted}{parent=normal text,use=block title alerted,bg=block title alerted.bg!50!bg}
     \setbeamercolor{block body example}{parent=normal text,use=block title example,bg=block title example.bg!50!bg}

     % or ...

     \setbeamercovered{transparent}
     % or whatever (possibly just delete it)
     \logo{\resizebox{!}{1.5cm}{\href{\basename{R}}{\includegraphics{image}}}}
   }

   \mode<handout>
   {
     \usetheme{PaloAlto}
     \usecolortheme{default}
     \setbeamercolor{sidebar}{bg=white, fg=black}
     \setbeamercolor{frametitle}{bg=white, fg=black}
     % or ...
     \setbeamercolor{logo}{bg=white}
     \setbeamercolor{block body}{parent=normal text,bg=white}
     \setbeamercolor{author in sidebar}{fg=black}
     \setbeamercolor{title in sidebar}{fg=black}
     \setbeamercovered{transparent}
     % or whatever (possibly just delete it)
     \logo{}
   }

   \usepackage{epsdice}
   \usepackage[latin1]{inputenc}
   \usepackage{graphics}
   \usepackage{amsmath,eepic,epic}

   \newcommand{\figslide}[3]{
   \begin{frame}
   \frametitle{#1}
     \begin{center}
     \resizebox{!}{2.7in}{\includegraphics{#2}}    
     \end{center}
   {#3}
   \end{frame}
   }

   \newcommand{\fighslide}[4]{
   \begin{frame}
   \frametitle{#1}
     \begin{center}
     \resizebox{!}{#4}{\includegraphics{#2}}    
     \end{center}
   {#3}
   \end{frame}
   }

   \newcommand{\figwref}[1]{
   \href{#1}{\tiny \tt #1}}

   \newcommand{\B}[1]{\beta_{#1}}
   \newcommand{\Bh}[1]{\widehat{\beta}_{#1}}
   \newcommand{\V}{\text{Var}}
   \newcommand{\Cov}{\text{Cov}}
   \newcommand{\Vh}{\widehat{\V}}
   \newcommand{\s}{\sigma}
   \newcommand{\sh}{\widehat{\sigma}}

   \newcommand{\argmax}[1]{\mathop{\text{argmax}}_{#1}}
   \newcommand{\argmin}[1]{\mathop{\text{argmin}}_{#1}}
   \newcommand{\Ee}{\mathbb{E}}
   \newcommand{\Pp}{\mathbb{P}}
   \newcommand{\real}{\mathbb{R}}
   \newcommand{\Ybar}{\overline{Y}}
   \newcommand{\Yh}{\widehat{Y}}
   \newcommand{\Xbar}{\overline{X}}
   \newcommand{\Tr}{\text{Tr}}


   \newcommand{\model}{{\cal M}}

   \newcommand{\figvskip}{-0.7in}
   \newcommand{\fighskip}{-0.3in}
   \newcommand{\figheight}{3.5in}

   \newcommand{\Rcode}[1]{{\bf \tt #1 }}
   \newcommand{\Rtcode}[1]{{\tiny \bf \tt #1 }}
   \newcommand{\Rscode}[1]{{\small \bf \tt #1 }}

   \newcommand{\RR}{{\tt R} \;}
   \newcommand{\basename}[1]{http://stats60.stanford.edu/#1}
   \newcommand{\dataname}[1]{\basename{data/#1}}
   \newcommand{\Rname}[1]{\basename{R/#1}}

   \newcommand{\mycolor}[1]{{\color{blue} #1}}
   \newcommand{\basehref}[2]{\href{\basename{#1}}{\mycolor{#2}}}
   \newcommand{\Rhref}[2]{\href{\basename{R/#1}}{\mycolor{#2}}}
   \newcommand{\datahref}[2]{\href{\dataname{#1}}{\mycolor{#2}}}
   \newcommand{\X}{\pmb{X}}
   \newcommand{\Y}{\pmb{Y}}
   \newcommand{\be}{\pmb{varepsilon}}
   \newcommand{\logit}{\text{logit}}


   \title{Statistics 60: Introduction to Statistical Methods}
   \subtitle{Review of some key concepts, part I} 
   \author{}% {Jonathan Taylor \\
   %Department of Statistics \\
   %Stanford University
   %}


   \begin{document}

   \begin{frame}
   \titlepage
   \end{frame}

   \part{Descriptive statistics}
   \frame{\partpage}

   %CODE
       % from matplotlib import rc
   % import pylab, numpy as np, sys
   % np.random.seed(0);import random; random.seed(0)
   % sys.path.append('/private/var/folders/dq/4_9bwd013ln6vvf_q110mwrh0000gn/T/tmpi2YQNH')
   % f=pylab.gcf(); f.set_size_inches(8.0,6.0)
   % datadir ='/private/var/folders/dq/4_9bwd013ln6vvf_q110mwrh0000gn/T/tmpi2YQNH/data'
   % import pylab
   % bins = [0,20,55,75,100]
   % count = [29,52,13,6]
   % PL_density(count, bins)
   % pylab.gca().set_xlim([0,100])
   % pylab.gca().set_ylim([0,2.])
   % pylab.gca().set_ylabel('Percent per year')
   % pylab.gca().set_xlabel('Age (years)')
   % 


   \begin{frame}
   \frametitle{Histogram}
   \begin{center}
   \resizebox{!}{2.7in}{\includegraphics{./images/inline/2733107db9.pdf}}    
   \end{center}

   \end{frame}

   %%%%%%%%%%%%%%%%%%%%%%%%%%%%%%%%%%%%%%%%%%%%%%%%%%%%%%%%%%%%

   \begin{frame} \frametitle{Numeric summaries of a data set / histogram}

   \begin{block}
   {Examples}
   \begin{itemize}
   \item Average:    $\bar{X} = \frac{1}{n} \sum_{i=1}^n X_i$.

     \item Standard deviation: $\text{SD}(X)=\sqrt{\frac{1}{n}\sum_{i=1}^n (X_i-\bar{X})^2}.$

       \item The {\em median} of a histogram is the number with half the area to
   the left and half the area to the right.

     \item The {\em first quartile} is the 25th percentile; the {\em third quartile} is the 75th percentile.

     \item The {\em inter-quartile range} is the difference between
     the third and first quartiles.

   \end{itemize}

   \end{block}
   \end{frame}

   %%%%%%%%%%%%%%%%%%%%%%%%%%%%%%%%%%%%%%%%%%%%%%%%%%%%%%%%%%%%

   \begin{frame} \frametitle{The normal curve}

   \begin{block}
   {Rules of thumb for SD}
   \begin{itemize}
   \item The area under the normal curve between -1 and +1 is about $68\%$.
   \item The area under the normal curve between -2 and +2 is about $95\%$.
   \item The area under the normal curve between -3 and +3 is about $99.7\%$.
   \end{itemize}
   \end{block}
   \end{frame}

   %%%%%%%%%%%%%%%%%%%%%%%%%%%%%%%%%%%%%%%%%%%%%%%%%%%%%%%%%%%%

   \begin{frame} \frametitle{The normal table: A-104}

   \begin{block}
   {Sample rows of the table look like}
   \begin{tabular}{rrr}
   $z$ & Height & Area \\ \hline
   0.70 & 31.23 & 51.61 \\
   1.00 & 24.20 & 68.27 \\
   2.00 & 5.40 & 95.45 \\
   \end{tabular}
   \end{block}
   \end{frame}

   %CODE
       % from matplotlib import rc
   % import pylab, numpy as np, sys
   % np.random.seed(0);import random; random.seed(0)
   % sys.path.append('/private/var/folders/dq/4_9bwd013ln6vvf_q110mwrh0000gn/T/tmpi2YQNH')
   % f=pylab.gcf(); f.set_size_inches(8.0,6.0)
   % datadir ='/private/var/folders/dq/4_9bwd013ln6vvf_q110mwrh0000gn/T/tmpi2YQNH/data'
   % import pylab, numpy as np
   % x = np.linspace(-4,4,101)
   % y = np.exp(-x**2/2) / np.sqrt(2*np.pi)
   % pylab.bar([-0.75,0.65], [31.23,31.23], width=0.1, color='red')
   % pylab.plot(x,y*100, linewidth=2)
   % pylab.gca().set_xlabel('standardized units')
   % pylab.gca().set_ylabel('% per standardized unit')
   % #pylab.gca().set_xlim([-2,4])
   % #pylab.gca().set_yticks([])
   % 


   \begin{frame}
   \frametitle{Using table A-104}
   \begin{center}
   \resizebox{!}{2.7in}{\includegraphics{./images/inline/5a118e8784.pdf}}    
   \end{center}
   The height at  $z= \pm 0.7$ is $31.23\%$ (per standardized unit)
   \end{frame}

   %CODE
       % from matplotlib import rc
   % import pylab, numpy as np, sys
   % np.random.seed(0);import random; random.seed(0)
   % sys.path.append('/private/var/folders/dq/4_9bwd013ln6vvf_q110mwrh0000gn/T/tmpi2YQNH')
   % f=pylab.gcf(); f.set_size_inches(8.0,6.0)
   % datadir ='/private/var/folders/dq/4_9bwd013ln6vvf_q110mwrh0000gn/T/tmpi2YQNH/data'
   % import pylab, numpy as np
   % x = np.linspace(-4,4,101)
   % y = np.exp(-x**2/2) / np.sqrt(2*np.pi)
   % x2 = np.linspace(-0.7,0.7,101)
   % y2 = np.exp(-x2**2/2) / np.sqrt(2*np.pi)
   % pylab.plot(x,y*100, linewidth=2)
   % xf, yf = pylab.poly_between(x2, 0*x2, y2*100)
   % pylab.fill(xf, yf, facecolor='r', hatch='\\', alpha=0.5)
   % pylab.gca().set_xlabel('standardized units')
   % pylab.gca().set_ylabel('% per standardized unit')
   % #pylab.gca().set_xlim([-2,4])
   % #pylab.gca().set_yticks([])
   % 


   \begin{frame}
   \frametitle{Using table A-104}
   \begin{center}
   \resizebox{!}{2.7in}{\includegraphics{./images/inline/92f14628eb.pdf}}    
   \end{center}
   The area between $z=-0.7$ and $z=0.7$ is $51.61\%$
   \end{frame}

   %%%%%%%%%%%%%%%%%%%%%%%%%%%%%%%%%%%%%%%%%%%%%%%%%%%%%%%%%%%%

   \begin{frame} \frametitle{Measurement}

   \begin{block}
   {Measurement model}

   \begin{itemize}
   \item    $
   \text{individual measurement} = \text{exact value} + \text{bias} + \text{chance error}
   $


   \item  $
   M = \mu + B + \epsilon
   $
   \end{itemize}

   \end{block}

   \begin{block}
     {Concepts}
     \begin{itemize}
     \item Treatment and control group.
       \item Randomized controlled experiment.
         \item Observational study.
           \item Double blind studies.
             \item Confounding.
     \end{itemize}
   \end{block}
   \end{frame}

   \part{Correlation and regression}
   \frame{\partpage}

   %CODE
       % from matplotlib import rc
   % import pylab, numpy as np, sys
   % np.random.seed(0);import random; random.seed(0)
   % sys.path.append('/private/var/folders/dq/4_9bwd013ln6vvf_q110mwrh0000gn/T/tmpi2YQNH')
   % f=pylab.gcf(); f.set_size_inches(8.0,6.0)
   % datadir ='/private/var/folders/dq/4_9bwd013ln6vvf_q110mwrh0000gn/T/tmpi2YQNH/data'
   % import matplotlib.mlab as ML
   % H = ML.csv2rec('%s/pearson_lee.csv' % datadir)
   % M = H['mother']
   % D = H['daughter']
   % pylab.scatter(M, D, c='red')
   % pylab.gca().set_xlabel("Mother's height (inches)")
   % pylab.gca().set_ylabel("Daughter's height (inches)")
   % 


   \begin{frame}
   \frametitle{Correlation \& scatterplots}
   \begin{center}
   \resizebox{!}{2.7in}{\includegraphics{./images/inline/a84b90b10a.pdf}}    
   \end{center}
   Correlation: a measure of {\em association}
   \end{frame}

   %%%%%%%%%%%%%%%%%%%%%%%%%%%%%%%%%%%%%%%%%%%%%%%%%%%%%%%%%%%%

   \begin{frame} \frametitle{Correlation}

   \begin{block}
   {Conceptual definition}

   \begin{itemize}
   \item A numerical summary of a scatterplot, i.e. a pair of lists.

   \item    If there is a strong association between two variables, then
   knowing one helps a lot in predicting the other. But when
   there is a weak association, information about one variable
   does not help much in guessing the other.

   \item The {\em correlation coefficient}, $r$ is a measure of the strength of this association.

   \item $r=+1$ if the variables are perfectly positively associated.

   \item $r=-1$ if the variables are perfectly negatively associated.
   \end{itemize}
   \end{block}
   \end{frame}

   %%%%%%%%%%%%%%%%%%%%%%%%%%%%%%%%%%%%%%%%%%%%%%%%%%%%%%%%%%%%

   \begin{frame} \frametitle{Correlation}

   \begin{block}
   {Computing $r$, the correlation coefficient}

   \begin{itemize}
   \item Given two lists, $X, Y$, convert them
   each to standardized units. Call these new lists $Z_X, Z_Y$.

   \item Make a new list $Z_{XY}$ whose entries are the products
   of the entries of $Z_X, Z_Y$.
   \item Then,
   $$
   r = \text{average}(Z_{XY}).
   $$
   \item Another way,
   $$
   r = \frac{\text{average(products $X, Y$)} - \text{average}(X) \times \text{average}(Y)}{\text{SD}(X) \times \text{SD}(Y)}.
   $$
   \end{itemize}
   \end{block}
   \end{frame}

   %CODE
       % from matplotlib import rc
   % import pylab, numpy as np, sys
   % np.random.seed(0);import random; random.seed(0)
   % sys.path.append('/private/var/folders/dq/4_9bwd013ln6vvf_q110mwrh0000gn/T/tmpi2YQNH')
   % f=pylab.gcf(); f.set_size_inches(8.0,6.0)
   % datadir ='/private/var/folders/dq/4_9bwd013ln6vvf_q110mwrh0000gn/T/tmpi2YQNH/data'
   % import matplotlib.mlab as ML
   % H = ML.csv2rec('%s/pearson_lee.csv' % datadir)
   % M = H['mother']
   % D = H['daughter']
   % pylab.scatter(M, D, c='red')
   % pylab.gca().set_xlabel("Mother's height (inches)")
   % pylab.gca().set_ylabel("Daughter's height (inches)")
   % pylab.title("r=%0.2f" % np.corrcoef([D,M])[0,1])
   % 


   \begin{frame}
   \frametitle{Correlation}
   \begin{center}
   \resizebox{!}{2.7in}{\includegraphics{./images/inline/5b695bd691.pdf}}    
   \end{center}
   Correlation is symmetric
   \end{frame}

   %CODE
       % from matplotlib import rc
   % import pylab, numpy as np, sys
   % np.random.seed(0);import random; random.seed(0)
   % sys.path.append('/private/var/folders/dq/4_9bwd013ln6vvf_q110mwrh0000gn/T/tmpi2YQNH')
   % f=pylab.gcf(); f.set_size_inches(8.0,6.0)
   % datadir ='/private/var/folders/dq/4_9bwd013ln6vvf_q110mwrh0000gn/T/tmpi2YQNH/data'
   % import matplotlib.mlab as ML
   % H = ML.csv2rec('%s/pearson_lee.csv' % datadir)
   % M = H['mother']
   % D = H['daughter']
   % pylab.scatter(D, M, c='red')
   % pylab.gca().set_ylabel("Mother's height (inches)")
   % pylab.gca().set_xlabel("Daughter's height (inches)")
   % pylab.title("r=%0.2f" % np.corrcoef([D,M])[0,1])
   % 


   \begin{frame}
   \frametitle{Correlation}
   \begin{center}
   \resizebox{!}{2.7in}{\includegraphics{./images/inline/e7bd5169f6.pdf}}    
   \end{center}
   Correlation is symmetric
   \end{frame}

   %CODE
       % from matplotlib import rc
   % import pylab, numpy as np, sys
   % np.random.seed(0);import random; random.seed(0)
   % sys.path.append('/private/var/folders/dq/4_9bwd013ln6vvf_q110mwrh0000gn/T/tmpi2YQNH')
   % f=pylab.gcf(); f.set_size_inches(8.0,6.0)
   % datadir ='/private/var/folders/dq/4_9bwd013ln6vvf_q110mwrh0000gn/T/tmpi2YQNH/data'
   % import matplotlib.mlab as ML
   % H = ML.csv2rec('%s/pearson_lee.csv' % datadir)
   % M = H['mother']
   % D = H['daughter']
   % pylab.scatter(D, M, c='red')
   % pylab.gca().set_ylabel("Mother's height (inches)")
   % pylab.gca().set_xlabel("Daughter's height (inches)")
   % pylab.title("r=%0.2f" % np.corrcoef([D,M])[0,1])
   % 


   \begin{frame}
   \frametitle{Correlation}
   \begin{center}
   \resizebox{!}{2.7in}{\includegraphics{./images/inline/e7bd5169f6.pdf}}    
   \end{center}
   Correlation is not causation
   \end{frame}

   %%%%%%%%%%%%%%%%%%%%%%%%%%%%%%%%%%%%%%%%%%%%%%%%%%%%%%%%%%%%

   \begin{frame} \frametitle{Regression}

   \begin{block}
   {Regression line}
   \begin{itemize}
   \item    The best fitting regression line
   passes through
   the point of averages and has slope
   $$
   \text{slope} = r(X,Y) \times \frac{SD(Y)}{SD(X)}.
   $$
   \item The intercept is
   $$
   \text{intercept} = \bar{Y} - \text{slope} \times \bar{X}.
   $$
   \end{itemize}

   \end{block}
   \end{frame}

   %CODE
       % from matplotlib import rc
   % import pylab, numpy as np, sys
   % np.random.seed(0);import random; random.seed(0)
   % sys.path.append('/private/var/folders/dq/4_9bwd013ln6vvf_q110mwrh0000gn/T/tmpi2YQNH')
   % f=pylab.gcf(); f.set_size_inches(8.0,6.0)
   % datadir ='/private/var/folders/dq/4_9bwd013ln6vvf_q110mwrh0000gn/T/tmpi2YQNH/data'
   % import matplotlib.mlab as ML
   % H = ML.csv2rec('%s/pearson_lee.csv' % datadir)
   % M = H['mother']
   % D = H['daughter']
   % pylab.scatter(M, D, c='red')
   % pylab.gca().set_xlabel("Mother's height (inches)")
   % pylab.gca().set_ylabel("Daughter's height (inches)")
   % Dbar = D.mean(); Dsd = np.sqrt(((D - Dbar)**2).mean())
   % Mbar = M.mean(); Msd = np.sqrt(((M - Mbar)**2).mean())
   % r = np.corrcoef([M, D])[0,1]
   % 
   % xf, yf = pylab.poly_between([67.5,68.5], [50,50], [75, 75])
   % g = (M < 68.5) * (M >= 67.5)
   % pylab.fill(xf, yf, facecolor='blue', alpha=0.4, hatch='/', label='_nolegend_')
   % 
   % pylab.plot([Mbar-3.5*Msd,Mbar,Mbar+3.5*Msd],
   %            [Dbar-3.5*Dsd,Dbar,Dbar+3.5*Dsd], 'b-', linewidth=3, label='SD line')
   % pylab.plot([Mbar-3.5*Msd,Mbar,Mbar+3.5*Msd],
   %            [Dbar-r*3.5*Dsd,Dbar,Dbar+r*3.5*Dsd], '-', linewidth=3, label='regression', color='black')
   % pylab.legend(['SD line', 'regression'])
   % 
   % def error(a,b):
   %     F = a*M+b
   %     return np.sqrt(np.sum((D-F)**2))
   % 
   % slope_SD = D.std() / M.std()
   % intercept_SD = D.mean() - slope_SD * M.mean()
   % 
   % slope_r = np.corrcoef([D,M])[0,1] * D.std() / M.std()
   % intercept_r = D.mean() - slope_r * M.mean()
   % 
   % pylab.title('Error(SD line)=%0.1f, Error(regression)=%0.1f' %
   %             (error(slope_SD, intercept_SD),
   %             error(slope_r, intercept_r)))
   % s = pylab.scatter([68],D[g].mean(), s=130, c='black', marker='^')
   % 


   \begin{frame}
   \frametitle{Regression}
   \begin{center}
   \resizebox{!}{2.7in}{\includegraphics{./images/inline/9d92f9b490.pdf}}    
   \end{center}
   The regression line is closer to the mean of 68 in group.
   \end{frame}

   %%%%%%%%%%%%%%%%%%%%%%%%%%%%%%%%%%%%%%%%%%%%%%%%%%%%%%%%%%%%

   \begin{frame} \frametitle{Regression}

   \begin{block}
   {Working with regressions}
   The following quantities are enough to do all
   regression problems
   \begin{itemize}
   \item $\text{average}(X), \text{SD}(X)$
   \item $\text{average}(X), \text{SD}(Y)$
   \item $r(X,Y)$
   \end{itemize}
   \end{block}

   \begin{block}
   {Example}
   It is believed that the more alcohol there is in a person's blood
   stream, the slower is that person's reaction times.
   An experiment with 10 subjects yields
   \begin{itemize}
   \item average amount of alcohol in blood $0.14\%$ with SD $0.04\%$
   \item average reaction time 0.42 seconds, SD 0.1 seconds
   \item correlation coefficient 0.8
   \end{itemize}
   \end{block}
   \end{frame}

   %%%%%%%%%%%%%%%%%%%%%%%%%%%%%%%%%%%%%%%%%%%%%%%%%%%%%%%%%%%%

   \begin{frame} \frametitle{Regression}

   \begin{block}
   {Question a)}
   \begin{description}
   \item[Q] Predict the reaction time of a person with an amount
   of alcohol of 0.22\%.
   \item[A1] One way is to first compute the slope, intercept
   $$
   \begin{aligned}
   \text{slope} &= \frac{0.8 \times 0.1}{0.04} = 2.0 \frac{\text{seconds}}{\%}     \\
   \text{intercept} &= 0.42 - 2. \times 0.14 = 0.14 \, \text{seconds}
   \end{aligned}
   $$
   So, the predicted time is
   $$
   2 \times 0.22 + 0.14 = 0.58 \, \text{seconds}
   $$
   \item[A2]
   $$
   0.42 + (0.22 - 0.14) \times 2. = 0.58 \, \text{seconds}
   $$
   \end{description}
   \end{block}
   \end{frame}

   %%%%%%%%%%%%%%%%%%%%%%%%%%%%%%%%%%%%%%%%%%%%%%%%%%%%%%%%%%%%

   \begin{frame} \frametitle{Regression}

   \begin{block}
   {Question b)}
   \begin{description}
   \item[Q] Find the regression line for regressing
   reaction time on alcohol level.
   \item[A] From part A1 on previous slide:
   the predicted reaction time as a function of alcohol level is
   $$
   0.14 + 2. \times \text{alcohol level in \%}
   $$
   \end{description}
   \end{block}
   \end{frame}

   %%%%%%%%%%%%%%%%%%%%%%%%%%%%%%%%%%%%%%%%%%%%%%%%%%%%%%%%%%%%

   \begin{frame} \frametitle{Regression}

   \begin{block}
   {Question c)}
   \begin{description}
   \item[Q] Predict the reaction time of a person
   whose alcohol level is at the 20th percentile. What percentile
   does that correspond to in reaction time?
   \item[A] The 20th percentile of blood alcohol is (using normal approximation)
   $$
   \begin{aligned}
   \lefteqn{0.14 + 0.04 \times \text{20th percentile of normal}} \\
   & \qquad = 0.14 + 0.04 \times (-0.84) \\
   & \qquad = 0.11
   \end{aligned}
   $$
   So, we predict a reaction time of
   $$
   0.14 + 2 \times 0.11 = 0.36
   $$
   \end{description}
   \end{block}
   \end{frame}

   %%%%%%%%%%%%%%%%%%%%%%%%%%%%%%%%%%%%%%%%%%%%%%%%%%%%%%%%%%%%

   \begin{frame} \frametitle{Regression}

   \begin{block}
   {Question c) continued}

   This is $(0.36-0.42)/0.1=-0.6$ in standardized units, so
   it corresponds to roughly the 30th percentile (from A-104)
   \end{block}

   \begin{block}
   {Question d)}
   \begin{description}
   \item[Q]    Predict the amount of alcohol a person has in
   her bloodstream if the reaction time is 0.37 seconds.
   \item[A] Using the A2 form from part a), we predict
   $$
   0.14 + 0.8 \times \frac{0.04}{0.1} (0.37-0.42) = 0.124 \%
   $$
   \end{description}
   \end{block}
   \end{frame}

   %%%%%%%%%%%%%%%%%%%%%%%%%%%%%%%%%%%%%%%%%%%%%%%%%%%%%%%%%%%%

   \begin{frame} \frametitle{Regression}

   \begin{block}
   {Regression fallacy}

   \begin{itemize}
   \item    Note that someone in the 20th percentile of alcohol
   level had predicted 30th percentile in reaction time.
   \item This is a general phenomenon, Galton referred to it as
   ``regression to mediocrity.''
   \end{itemize}

   \end{block}

   \begin{block}
   {Test-retest version of regression fallacy}
   In a test-retest situation, usually the bottom
   group on the first test will show some improvement,
   and the top group will fall back.
   \end{block}
   \end{frame}

   %%%%%%%%%%%%%%%%%%%%%%%%%%%%%%%%%%%%%%%%%%%%%%%%%%%%%%%%%%%%

   \begin{frame} \frametitle{Regression}

   \begin{block}
   {\text{r.m.s.} of regression}
   \begin{itemize}
   \item Regression line is chosen to minimize
   the r.m.s. of the residuals of all ines
   $$
   \begin{aligned}
   \text{r.m.s.}(\text{regression $Y$ on $X$}) &= \text{r.m.s.(residuals)} \\
   &= \sqrt{\text{average}(\text{residuals}^2)} \\
   \end{aligned}
   $$
   \item This r.m.s. is always less than the SD of
   the dependent variable ($Y$) alone
   $$
   \text{r.m.s.}(\text{regression $Y$ on $X$}) = \sqrt{1-r^2} \times \text{SD}(Y)
   $$
   \item Useful if data cloud is {\em football shaped}.
   \end{itemize}
   \end{block}
   \end{frame}

   %%%%%%%%%%%%%%%%%%%%%%%%%%%%%%%%%%%%%%%%%%%%%%%%%%%%%%%%%%%%

   \begin{frame} \frametitle{Regression}

   \begin{block}
   {Example: using regression r.m.s. in vertical strips}
   \begin{itemize}
   \item Given the following
     $$
     \begin{aligned}
       \text{average(mother)} &= 62.4\\
       \text{SD(mother)} &= 2.3 \\
       \text{average(daughter)} &= 63.8 \\
       \text{SD(daughter)} &= 2.6 \\
       \text{r(mother, daughter)} &= 0.49
     \end{aligned}
     $$
   Of mothers of height 66in, what percentage
   of their daughters will have height above 67in?

   \end{itemize}
   \end{block}
   \end{frame}

   %%%%%%%%%%%%%%%%%%%%%%%%%%%%%%%%%%%%%%%%%%%%%%%%%%%%%%%%%%%%

   \begin{frame} \frametitle{Regression}

   \begin{block}
   {Answer}
   \begin{itemize}
   \item Slope of regression line is
   $$
   \text{slope} = 0.49 \times \frac{2.6}{2.3} = 0.54
   $$
   \item The average height of daughters of
   mothers of height 66in is
   $$
   63.8 + 0.54 \times (66 - 62.4) = 65.7
   $$

   \item The SD is taken to be r.m.s. of regression
   $$
   0.87 \times 2.6 = 2.3.
   $$


     \item 67 in corresponds to $(67-65.7)/2.3=0.6$ standardized units.

     \item From A-104, the percentage is roughly 27\%.

   \end{itemize}
   \end{block}
   \end{frame}

   \part{Probability}
   \frame{\partpage}

   %%%%%%%%%%%%%%%%%%%%%%%%%%%%%%%%%%%%%%%%%%%%%%%%%%%%%%%%%%%%

   \begin{frame} \frametitle{Probability}

   \begin{block}
   {Multiplication rule}
   \begin{itemize}
   \item $$P(A \cap B) = P(B|A) \times P(A).
    $$
   \item Independence: two events $A$ and $B$ are independent if
   $$P(A \cap B) = P(B) \times P(A).$$
   \end{itemize}
   \end{block}
   \end{frame}

   %%%%%%%%%%%%%%%%%%%%%%%%%%%%%%%%%%%%%%%%%%%%%%%%%%%%%%%%%%%%

   \begin{frame} \frametitle{Probability}

     \begin{block}
   {Addition rule}
   \begin{itemize}
   \item    If the events $E_1, E_2$ are mutually exclusive, then
   $$
   P(\text{$E_1$ or $E_2$}) = P(E_1) + P(E_2).
   $$
   \item This rule works for more than two: if $[E_1, \dots, E_n]$
     are mutually exclusive, then
   $$
   P(\text{$E_1$ or $E_2$ or \dots or $E_n$}) = \sum_{i=1}^n P(E_i).
   $$
   \item If not mutually exclusive
   $$
   P(E_1 \cup E_2) = P(E_1) + P(E_2) - P(E_1 \cap E_2)
   $$

   \end{itemize}
   \end{block}
   \end{frame}

   %CODE
       % from matplotlib import rc
   % import pylab, numpy as np, sys
   % np.random.seed(0);import random; random.seed(0)
   % sys.path.append('/private/var/folders/dq/4_9bwd013ln6vvf_q110mwrh0000gn/T/tmpi2YQNH')
   % f=pylab.gcf(); f.set_size_inches(6.0,6.0)
   % datadir ='/private/var/folders/dq/4_9bwd013ln6vvf_q110mwrh0000gn/T/tmpi2YQNH/data'
   % import numpy as np, pylab, matplotlib
   % 
   % import matplotlib
   % 
   % rc('text', usetex=True)
   % 
   % cir = matplotlib.patches.Circle
   % 
   % a = pylab.gca()
   % # add a circle
   % E1 = cir((0.5,0.5), 0.4,ec="black", facecolor='yellow',lw=2, alpha=0.4)
   % E2 = cir((0.2,0.2), 0.4,ec="black", facecolor='blue',lw=2, alpha=0.4)
   % a.add_patch(E1)
   % a.add_patch(E2)
   % a.set_xticks([])
   % a.set_yticks([])
   % a.set_xlim([-0.3,1])
   % a.set_ylim([-0.3,1])
   % 


   \begin{frame}
   \frametitle{Non-mutually exclusive events}
   \begin{center}
   \resizebox{!}{2.7in}{\includegraphics{./images/inline/c1b6d1a77a.pdf}}    
   \end{center}

   \end{frame}

   %%%%%%%%%%%%%%%%%%%%%%%%%%%%%%%%%%%%%%%%%%%%%%%%%%%%%%%%%%%%

   \begin{frame} \frametitle{Probability}

   \begin{block}
   {What are the chances the sum will be greater than or equal to 7?}
   \begin{table}
     \centering
   \begin{tabular}{cccccc}
   \epsdice{1} \, \epsdice[black]{1} & \epsdice{1} \, \epsdice[black]{2} & \epsdice{1} \, \epsdice[black]{3} & \epsdice{1} \, \epsdice[black]{4} & \epsdice{1} \, \epsdice[black]{5} & {\color{red} \fbox{\epsdice{1} \, \epsdice[black]{6}}} \\
   \epsdice{2} \, \epsdice[black]{1} & \epsdice{2} \, \epsdice[black]{2} & \epsdice{2} \, \epsdice[black]{3} & \epsdice{2} \, \epsdice[black]{4} & {\color{red} \fbox{\epsdice{2} \, \epsdice[black]{5}}} & {\color{red} \fbox{\epsdice{2} \, \epsdice[black]{6}}} \\
   \epsdice{3} \, \epsdice[black]{1} & \epsdice{3} \, \epsdice[black]{2} & \epsdice{3} \, \epsdice[black]{3} & {\color{red} \fbox{\epsdice{3} \, \epsdice[black]{4}}} & {\color{red} \fbox{\epsdice{3} \, \epsdice[black]{5}}} & {\color{red} \fbox{\epsdice{3} \, \epsdice[black]{6}}} \\
   \epsdice{4} \, \epsdice[black]{1} & \epsdice{4} \, \epsdice[black]{2} & {\color{red} \fbox{\epsdice{4} \, \epsdice[black]{3}}} & {\color{red} \fbox{\epsdice{4} \, \epsdice[black]{4}}} & {\color{red} \fbox{\epsdice{4} \, \epsdice[black]{5}}} & {\color{red} \fbox{\epsdice{4} \, \epsdice[black]{6}}} \\
   \epsdice{5} \, \epsdice[black]{1} & {\color{red} \fbox{\epsdice{5} \, \epsdice[black]{2}}} & {\color{red} \fbox{\epsdice{5} \, \epsdice[black]{3}}} & {\color{red} \fbox{\epsdice{5} \, \epsdice[black]{4}}} & {\color{red} \fbox{\epsdice{5} \, \epsdice[black]{5}}} & {\color{red} \fbox{\epsdice{5} \, \epsdice[black]{6}}} \\
   {\color{red} \fbox{\epsdice{6} \, \epsdice[black]{1}}} & {\color{red} \fbox{\epsdice{6} \, \epsdice[black]{2}}} & {\color{red} \fbox{\epsdice{6} \, \epsdice[black]{3}}} & {\color{red} \fbox{\epsdice{6} \, \epsdice[black]{4}}} & {\color{red} \fbox{\epsdice{6} \, \epsdice[black]{5}}} & {\color{red} \fbox{\epsdice{6} \, \epsdice[black]{6}}} \\
   \end{tabular}
   \end{table}
   There are 21 outcomes whose sum is greater than or equal to 7. Therefore, the chances are $\frac{21}{36}=\frac{7}{12}$.
   \end{block}
   \end{frame}

   %%%%%%%%%%%%%%%%%%%%%%%%%%%%%%%%%%%%%%%%%%%%%%%%%%%%%%%%%%%%

   \begin{frame} \frametitle{Probability}

   \begin{block}
   {Complement of an event}
   \begin{itemize}
   \item Formally, the ``opposite'' rule is the rule
   of {\em complements}.
   \item We write the complement of an event $E$ as $E^c$
   $$
   P(\text{not} \, E) = P(E^c).
   $$
   \item The rule of {\em complements} says
   $$
   P(E^c) = 1 - P(E)
   $$
   \end{itemize}
   \end{block}
   \end{frame}

   %%%%%%%%%%%%%%%%%%%%%%%%%%%%%%%%%%%%%%%%%%%%%%%%%%%%%%%%%%%%

   \begin{frame} \frametitle{Probability}

   \begin{block}
   {Bayes' theorem}
   \begin{itemize}
   \item Given two events $A$ and $B$
   $$
   \begin{aligned}
   P(A|B) &= \frac{P(B \, \text{and} \,  A)}{P(B)} \\
   &= \frac{P(B|A)\times P(A)}{P(B)} \\
   &= \frac{P(B|A) \times P(A)}{P(B|A) \times P(A) + P(B|A^c) \times P(A^c)     } \\
   \end{aligned}
   $$
   \item A consequence of the multiplication rule \dots

   \end{itemize}
   \end{block}
   \end{frame}

   %%%%%%%%%%%%%%%%%%%%%%%%%%%%%%%%%%%%%%%%%%%%%%%%%%%%%%%%%%%%

   \begin{frame} \frametitle{Binomial formula}

   \begin{block}
   {Drawing $k$ balls out of $n$, ignoring order}

   The number of ways of drawing $k$ balls without replacement (ignoring order)
   from $n$ is
   $$
   \binom{n}{k} = \frac{n!}{k! \times (n-k)!}
   $$
   \end{block}
   \end{frame}

   %%%%%%%%%%%%%%%%%%%%%%%%%%%%%%%%%%%%%%%%%%%%%%%%%%%%%%%%%%%%

   \begin{frame} \frametitle{Binomial formula}

   \begin{block}
   {Example}

   \begin{description}
   \item[Q]    When flipping a coin 10 times, how many outcomes are there
   with 7 heads?

   \item[A] We can represent this as drawing 7 out of a possible
   10 slots for the heads, without order. There are
   $$
   \binom{10}{7} = \frac{10 \times 9 \times 8}{3 \times 2 \times 1} = 120 \
   \text{outcomes}   $$
   \end{description}

   \end{block}

   \begin{block}
   {Binomial formula for computing probabilities}
   When performing $n$ independent trials, each
   with probability of success $p$, the probability
   of observing exactly $k$ successes is
   $$
   \binom{n}{ k} { p^k} {(1-p)^ {n-k}}
   $$
   \end{block}
   \end{frame}

   %%%%%%%%%%%%%%%%%%%%%%%%%%%%%%%%%%%%%%%%%%%%%%%%%%%%%%%%%%%%

   \begin{frame} 

   \end{frame}

   \end{document}
