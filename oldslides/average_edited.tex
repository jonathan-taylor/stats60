   \documentclass[handout]{beamer}



   \mode<presentation>
   {
     \usetheme{PaloAlto}
   \setbeamertemplate{footline}[page number]

     \setbeamercolor{sidebar}{bg=white, fg=black}
     \setbeamercolor{frametitle}{bg=white, fg=black}
     % or ...
     \setbeamercolor{logo}{bg=white}
     \setbeamercolor{block body}{parent=normal text,bg=white}
     \setbeamercolor{author in sidebar}{fg=black}
     \setbeamercolor{title in sidebar}{fg=black}


     \setbeamercolor*{block title}{use=structure,fg=structure.fg,bg=structure.fg!20!bg}
     \setbeamercolor*{block title alerted}{use=alerted text,fg=alerted text.fg,bg=alerted text.fg!20!bg}
     \setbeamercolor*{block title example}{use=example text,fg=example text.fg,bg=example text.fg!20!bg}


     \setbeamercolor{block body}{parent=normal text,use=block title,bg=block title.bg!50!bg}
     \setbeamercolor{block body alerted}{parent=normal text,use=block title alerted,bg=block title alerted.bg!50!bg}
     \setbeamercolor{block body example}{parent=normal text,use=block title example,bg=block title example.bg!50!bg}

     % or ...

     \setbeamercovered{transparent}
     % or whatever (possibly just delete it)
     \logo{\resizebox{!}{1.5cm}{\href{\basename{R}}{\includegraphics{image}}}}
   }

   \mode<handout>
   {
     \usetheme{PaloAlto}
     \usecolortheme{default}
     \setbeamercolor{sidebar}{bg=white, fg=black}
     \setbeamercolor{frametitle}{bg=white, fg=black}
     % or ...
     \setbeamercolor{logo}{bg=white}
     \setbeamercolor{block body}{parent=normal text,bg=white}
     \setbeamercolor{author in sidebar}{fg=black}
     \setbeamercolor{title in sidebar}{fg=black}
     \setbeamercovered{transparent}
     % or whatever (possibly just delete it)
     \logo{}
   }

   \usepackage{epsdice}
   \usepackage[latin1]{inputenc}
   \usepackage{graphics}
   \usepackage{amsmath,eepic,epic}

   \newcommand{\figslide}[3]{
   \begin{frame}
   \frametitle{#1}
     \begin{center}
     \resizebox{!}{2.7in}{\includegraphics{#2}}    
     \end{center}
   {#3}
   \end{frame}
   }

   \newcommand{\fighslide}[4]{
   \begin{frame}
   \frametitle{#1}
     \begin{center}
     \resizebox{!}{#4}{\includegraphics{#2}}    
     \end{center}
   {#3}
   \end{frame}
   }

   \newcommand{\figwref}[1]{
   \href{#1}{\tiny \tt #1}}

   \newcommand{\B}[1]{\beta_{#1}}
   \newcommand{\Bh}[1]{\widehat{\beta}_{#1}}
   \newcommand{\V}{\text{Var}}
   \newcommand{\Cov}{\text{Cov}}
   \newcommand{\Vh}{\widehat{\V}}
   \newcommand{\s}{\sigma}
   \newcommand{\sh}{\widehat{\sigma}}

   \newcommand{\argmax}[1]{\mathop{\text{argmax}}_{#1}}
   \newcommand{\argmin}[1]{\mathop{\text{argmin}}_{#1}}
   \newcommand{\Ee}{\mathbb{E}}
   \newcommand{\Pp}{\mathbb{P}}
   \newcommand{\real}{\mathbb{R}}
   \newcommand{\Ybar}{\overline{Y}}
   \newcommand{\Yh}{\widehat{Y}}
   \newcommand{\Xbar}{\overline{X}}
   \newcommand{\Tr}{\text{Tr}}


   \newcommand{\model}{{\cal M}}

   \newcommand{\figvskip}{-0.7in}
   \newcommand{\fighskip}{-0.3in}
   \newcommand{\figheight}{3.5in}

   \newcommand{\Rcode}[1]{{\bf \tt #1 }}
   \newcommand{\Rtcode}[1]{{\tiny \bf \tt #1 }}
   \newcommand{\Rscode}[1]{{\small \bf \tt #1 }}

   \newcommand{\RR}{{\tt R} \;}
   \newcommand{\basename}[1]{http://stats60.stanford.edu/#1}
   \newcommand{\dataname}[1]{\basename{data/#1}}
   \newcommand{\Rname}[1]{\basename{R/#1}}

   \newcommand{\mycolor}[1]{{\color{blue} #1}}
   \newcommand{\basehref}[2]{\href{\basename{#1}}{\mycolor{#2}}}
   \newcommand{\Rhref}[2]{\href{\basename{R/#1}}{\mycolor{#2}}}
   \newcommand{\datahref}[2]{\href{\dataname{#1}}{\mycolor{#2}}}
   \newcommand{\X}{\pmb{X}}
   \newcommand{\Y}{\pmb{Y}}
   \newcommand{\be}{\pmb{varepsilon}}
   \newcommand{\logit}{\text{logit}}


   \title{Statistics 60: Introduction to Statistical Methods}
   \subtitle{Chapter 4: Average \& Standard Deviation} 
   \author{}% {Jonathan Taylor \\
   %Department of Statistics \\
   %Stanford University
   %}


   \begin{document}

   \begin{frame}
   \titlepage
   \end{frame}

   %%%%%%%%%%%%%%%%%%%%%%%%%%%%%%%%%%%%%%%%%%%%%%%%%%%%%%%%%%%%

   \begin{frame} \frametitle{Average}

   \begin{block}
   {Definition}

   The average of a list of numbers equals their sum, divided by how
   many there are. Average is also called the {\em mean}.

   \end{block}

   \begin{block}
   {Example: [1,4,6,7,8]}

   $$
   \text{Average} = \frac{1+4+6+7+8}{5} = \frac{26}{5} = 5.2
   $$

   \end{block}
   \end{frame}

   %%%%%%%%%%%%%%%%%%%%%%%%%%%%%%%%%%%%%%%%%%%%%%%%%%%%%%%%%%%%

   \begin{frame} \frametitle{Average}

   \begin{block}
   {$\Sigma$ notation}

   \begin{itemize}
   \item Call our list $X=[X_1, \dots, X_n]$.

   \item We often write the sum
   $$X_1 + X_2 + \dots + X_n = \sum_{i=1}^n X_i.$$

   \item The mean of a list $X = [X_1, \dots, X_n]$ is often written as
   $$
   \bar{X} = \frac{1}{n} \sum_{i=1}^n X_i.
   $$
   \end{itemize}
   \end{block}
   \end{frame}

   %CODE
       % from matplotlib import rc
   % import pylab, numpy as np, sys
   % np.random.seed(0);import random; random.seed(0)
   % sys.path.append('/private/var/folders/dq/4_9bwd013ln6vvf_q110mwrh0000gn/T/tmpi3DZIt')
   % f=pylab.gcf(); f.set_size_inches(8.0,6.0)
   % datadir ='/private/var/folders/dq/4_9bwd013ln6vvf_q110mwrh0000gn/T/tmpi3DZIt/data'
   % import pylab
   % bins = []; count = []
   % data = np.array([1,4,6,7,8])
   % for n in data:
   %     bins += [n-0.05,n+0.05]
   %     count += [1,0]
   % count = count[:-1]
   % mean = data.mean()
   % PL_density(count, bins)
   % pylab.arrow(mean, 10, 0, -10,color='r')
   % pylab.gca().set_xlim([0,11.])
   % pylab.gca().set_yticks([])
   % pylab.gca().set_ylim([0,220.])
   % 


   \begin{frame}
   \frametitle{Average}
   \begin{center}
   \resizebox{!}{2.7in}{\includegraphics{./images/inline/b4ac0e1437.pdf}}    
   \end{center}

   \end{frame}

   %CODE
       % from matplotlib import rc
   % import pylab, numpy as np, sys
   % np.random.seed(0);import random; random.seed(0)
   % sys.path.append('/private/var/folders/dq/4_9bwd013ln6vvf_q110mwrh0000gn/T/tmpi3DZIt')
   % f=pylab.gcf(); f.set_size_inches(8.0,6.0)
   % datadir ='/private/var/folders/dq/4_9bwd013ln6vvf_q110mwrh0000gn/T/tmpi3DZIt/data'
   % import pylab
   % bins = []; count = []
   % data = np.array([1,4])
   % bins = [0.95,1.05,3.95,4.05]
   % count = [2,0,2]
   % mean = (2*1+2*4) / 4.
   % PL_density(count, bins)
   % pylab.arrow(mean, 60, 0, -60,color='r')
   % pylab.gca().set_xlim([0,5.])
   % pylab.gca().set_yticks([])
   % pylab.gca().set_ylim([0,520.])
   % 


   \begin{frame}
   \frametitle{Average of [1,4]}
   \begin{center}
   \resizebox{!}{2.7in}{\includegraphics{./images/inline/85b1086b74.pdf}}    
   \end{center}

   \end{frame}

   %CODE
       % from matplotlib import rc
   % import pylab, numpy as np, sys
   % np.random.seed(0);import random; random.seed(0)
   % sys.path.append('/private/var/folders/dq/4_9bwd013ln6vvf_q110mwrh0000gn/T/tmpi3DZIt')
   % f=pylab.gcf(); f.set_size_inches(8.0,6.0)
   % datadir ='/private/var/folders/dq/4_9bwd013ln6vvf_q110mwrh0000gn/T/tmpi3DZIt/data'
   % import pylab
   % bins = []; count = []
   % data = np.array([1,4])
   % bins = [0.95,1.05,3.95,4.05]
   % count = [4,0,2]
   % mean = (4*1+2*4) / 6.
   % PL_density(count, bins)
   % pylab.arrow(mean, 70, 0, -70,color='r')
   % pylab.gca().set_xlim([0,5.])
   % pylab.gca().set_yticks([])
   % pylab.gca().set_ylim([0,750.])
   % 


   \begin{frame}
   \frametitle{Average of [1,1,1,1,4,4]}
   \begin{center}
   \resizebox{!}{2.7in}{\includegraphics{./images/inline/25cd01ed2e.pdf}}    
   \end{center}

   \end{frame}

   %%%%%%%%%%%%%%%%%%%%%%%%%%%%%%%%%%%%%%%%%%%%%%%%%%%%%%%%%%%%

   \begin{frame} \frametitle{Average of a list}

   \begin{block}
   {Average balances the list}
   The sum of the deviations from average is always zero.

   \end{block}

   \begin{block}
   {Example: [1,1,1,1,4,4]}


   \begin{itemize}
   \item The average is 2.

   \item Deviations from average is $[-1,-1,-1,-1,2,2]$.

   \item Sum of deviations from average: 0. This is always true.

   \item In $\Sigma$ notation
   $$
   \sum_{i=1}^n (X_i - \bar{X}) = 0.
   $$

   \end{itemize}
   \end{block}
   \end{frame}

   %%%%%%%%%%%%%%%%%%%%%%%%%%%%%%%%%%%%%%%%%%%%%%%%%%%%%%%%%%%%

   \begin{frame} \frametitle{Average of a list}

   \begin{block}
   {Average balances the list}
   The sum of the deviations from average is always zero.

   \end{block}

   \begin{block}
   {Example: [1,1,1,1,4,4]}


   \begin{itemize}
   \item With repeats, we can think of having a ``weight'' of 4 at 1 and 2 at 4.

   \item Deviations from average is $-1=(1-2)$ at $2=(4-2)$ at 4.

   \item Weighted sum of deviations from average: $4 \times (-1) + 2 \times 2 = 0$.

   \item This is how to compute the average of a histogram.

   \end{itemize}
   \end{block}
   \end{frame}

   %CODE
       % from matplotlib import rc
   % import pylab, numpy as np, sys
   % np.random.seed(0);import random; random.seed(0)
   % sys.path.append('/private/var/folders/dq/4_9bwd013ln6vvf_q110mwrh0000gn/T/tmpi3DZIt')
   % f=pylab.gcf(); f.set_size_inches(8.0,6.0)
   % datadir ='/private/var/folders/dq/4_9bwd013ln6vvf_q110mwrh0000gn/T/tmpi3DZIt/data'
   % import pylab
   % bins = []; count = []
   % data = np.array([1,4])
   % bins = np.linspace(0,1,21)
   % X = np.random.beta(2,6,10000)
   % count = np.histogram(X, bins)[0]
   % mean = X.mean()
   % PL_density(count, bins)
   % pylab.arrow(mean, 70, 0, -70,color='r', linewidth=3)
   % #   pylab.gca().set_xlim([0,1.])
   % pylab.gca().set_yticks([])
   % #   pylab.gca().set_ylim([0,750.])
   % 


   \begin{frame}
   \frametitle{Average of a histogram}
   \begin{center}
   \resizebox{!}{2.7in}{\includegraphics{./images/inline/ca04ecadff.pdf}}    
   \end{center}
   A histogram is balanced at the average
   \end{frame}

   %%%%%%%%%%%%%%%%%%%%%%%%%%%%%%%%%%%%%%%%%%%%%%%%%%%%%%%%%%%%

   \begin{frame} \frametitle{Median}

   \begin{block}
   {Another measure of center}
   The {\em median} of a histogram is the number with half the area to
   the left and half the area to the right.

   \end{block}

   \begin{block}
   {Median of a list of numbers}

   \begin{description}
   \item[Step 1:] Sort the numbers for smallest to largest.

   \item[Step 2a:] If the length of the list is odd, the median is the middle
   entry of the sorted values.

   \item[Step 2b:] If the length of the list is even, the median is the average
   of the two middle entries.

   \end{description}
   \end{block}
   \end{frame}

   %%%%%%%%%%%%%%%%%%%%%%%%%%%%%%%%%%%%%%%%%%%%%%%%%%%%%%%%%%%%

   \begin{frame} \frametitle{Median}

   \begin{block}
   {Example: median of $[1,4,2,9,8]$}

   \begin{description}
   \item[Step 1:] The sorted values are $[1,2,4,8,9]$.

   \item[Step 2a:] Since the length of the list is 5, the median is the
   middle entry of the sorted values. The median is 4.

   \end{description}
   \end{block}

   \begin{block}
   {Example: median of $[1,11,3,7,8,3]$}

   \begin{description}
   \item[Step 1:] The sorted values are $[1,3,3,7,8,11]$.

   \item[Step 2a:] Since the length of the list is 6, the median is the
   average of the middle entries. The median is $(3+7)/2=5$.

   \end{description}
   \end{block}
   \end{frame}

   %%%%%%%%%%%%%%%%%%%%%%%%%%%%%%%%%%%%%%%%%%%%%%%%%%%%%%%%%%%%

   \begin{frame} \frametitle{Median and average}

   \begin{block}
   {Examples}

   \begin{itemize}
   \item Q: What is the mean of $[3,7,4,11,5]$? The median?

   \item A: The mean is 6. The median is 5.


   \end{itemize}
   \end{block}
   \end{frame}

   %%%%%%%%%%%%%%%%%%%%%%%%%%%%%%%%%%%%%%%%%%%%%%%%%%%%%%%%%%%%

   \begin{frame} \frametitle{Median and average}

   \begin{block}
   {Examples}

   \begin{itemize}
   \item Q: What is the mean of $[3,37,4,41,5]$? The median?

   \item A: The mean is 18. The median is 5.

   \end{itemize}

   \end{block}

   \begin{block}
     {What's the difference?}
   The median is less sensitive to changes away from the center than
   the mean is.
   \end{block}
   \end{frame}

   %%%%%%%%%%%%%%%%%%%%%%%%%%%%%%%%%%%%%%%%%%%%%%%%%%%%%%%%%%%%

   \begin{frame} \frametitle{Symmetric histogram}

   \begin{figure}
   \centering
   \resizebox{!}{2.0in}{\includegraphics{figs/descriptive/symmetric}}
   \end{figure}

   \begin{center}
   Average $=$ Median
   \end{center}
   \end{frame}

   %%%%%%%%%%%%%%%%%%%%%%%%%%%%%%%%%%%%%%%%%%%%%%%%%%%%%%%%%%%%

   \begin{frame} \frametitle{Skewed left histogram}

   \begin{figure}
   \centering
   \resizebox{!}{2.0in}{\includegraphics{figs/descriptive/skewleft}}
   \end{figure}

   \begin{center}
   Average $<$ Median
   \end{center}
   \end{frame}

   %%%%%%%%%%%%%%%%%%%%%%%%%%%%%%%%%%%%%%%%%%%%%%%%%%%%%%%%%%%%

   \begin{frame} \frametitle{Skewed right histogram}

   \begin{figure}
   \centering
   \resizebox{!}{1.8in}{\includegraphics{figs/descriptive/skewright}}

   \end{figure}
   \begin{center}
   Average $>$ Median
   \end{center}
   \end{frame}

   %CODE
       % from matplotlib import rc
   % import pylab, numpy as np, sys
   % np.random.seed(0);import random; random.seed(0)
   % sys.path.append('/private/var/folders/dq/4_9bwd013ln6vvf_q110mwrh0000gn/T/tmpi3DZIt')
   % f=pylab.gcf(); f.set_size_inches(8.0,6.0)
   % datadir ='/private/var/folders/dq/4_9bwd013ln6vvf_q110mwrh0000gn/T/tmpi3DZIt/data'
   % import pylab, numpy as np
   % bins = np.linspace(-2,4,51)
   % X = np.random.standard_normal(5000) + 1
   % Y = (X - 1) * 0.5 + 1
   % pylab.hist(X, bins)
   % pylab.gca().set_yticks([])
   % pylab.gca().set_xlim([-2,4])
   % 


   \begin{frame}
   \frametitle{Scale or spread of a data set}
   \begin{center}
   \resizebox{!}{2.7in}{\includegraphics{./images/inline/38a37538b0.pdf}}    
   \end{center}

   \end{frame}

   %CODE
       % from matplotlib import rc
   % import pylab, numpy as np, sys
   % np.random.seed(0);import random; random.seed(0)
   % sys.path.append('/private/var/folders/dq/4_9bwd013ln6vvf_q110mwrh0000gn/T/tmpi3DZIt')
   % f=pylab.gcf(); f.set_size_inches(8.0,6.0)
   % datadir ='/private/var/folders/dq/4_9bwd013ln6vvf_q110mwrh0000gn/T/tmpi3DZIt/data'
   % import pylab, numpy as np
   % bins = np.linspace(-2,4,51)
   % X = np.random.standard_normal(5000) + 1
   % Y = (X - 1) * 0.5 + 1
   % pylab.hist(Y, bins)
   % pylab.gca().set_xlim([-2,4])
   % pylab.gca().set_yticks([])
   % 


   \begin{frame}
   \frametitle{Scale or spread of a data set}
   \begin{center}
   \resizebox{!}{2.7in}{\includegraphics{./images/inline/fe22e771fa.pdf}}    
   \end{center}
   Average of both histograms is the same but spread is different.
   \end{frame}

   %%%%%%%%%%%%%%%%%%%%%%%%%%%%%%%%%%%%%%%%%%%%%%%%%%%%%%%%%%%%

   \begin{frame} \frametitle{Standard deviation}

   \begin{block}
   {Conceptual definition}

   \begin{itemize}
   \item    The SD says how far numbers on a list are away from their
   average.
   \item Its units are in the original units if the list.

   \item Most entries on the list will be somehere around one SD
   away from the average. Very few will be more than two or three SDs away.
   \end{itemize}

   \end{block}
   \end{frame}

   %%%%%%%%%%%%%%%%%%%%%%%%%%%%%%%%%%%%%%%%%%%%%%%%%%%%%%%%%%%%

   \begin{frame} \frametitle{Standard deviation}

   \begin{block}
   {Rules of thumb}

   \begin{itemize}
   \item  Roughly 68 \% of the entries on a list are within one SD of the average.
   \item  Roughly 95 \% of the entries on a list are within two SDs of the average.
   \item These rules hold for many data sets, not all.
   \end{itemize}
   \end{block}
   \end{frame}

   %CODE
       % from matplotlib import rc
   % import pylab, numpy as np, sys
   % np.random.seed(0);import random; random.seed(0)
   % sys.path.append('/private/var/folders/dq/4_9bwd013ln6vvf_q110mwrh0000gn/T/tmpi3DZIt')
   % f=pylab.gcf(); f.set_size_inches(8.0,6.0)
   % datadir ='/private/var/folders/dq/4_9bwd013ln6vvf_q110mwrh0000gn/T/tmpi3DZIt/data'
   % import pylab, numpy as np
   % X = np.random.standard_normal(5000) + 1
   % Y = (X - 1) * 0.5 + 1 + np.random.uniform(size=X.shape[0]) * 2.
   % mean = Y.mean()
   % SD = np.sqrt(((Y - Y.mean())**2).mean())
   % pylab.hist(Y, bins=30, alpha=0.5)
   % pylab.arrow(mean, 70, 0, -70,color='r', linewidth=3)
   % pylab.arrow(mean-SD, 70, 0, -70,color='y', linewidth=3)
   % pylab.arrow(mean+SD, 70, 0, -70,color='y', linewidth=3)
   % pylab.arrow(mean-2*SD, 70, 0, -70,color='y', linewidth=3)
   % pylab.arrow(mean+2*SD, 70, 0, -70,color='y', linewidth=3)
   % standY = (Y - mean) / SD
   % within1 = (np.fabs(standY) <= 1).sum() * 1. / Y.shape[0] * 100
   % within2 = (np.fabs(standY) <= 2).sum() * 1. / Y.shape[0] * 100
   % pylab.title('Within 1 SD %d %%, Within 2 SD %d %%' % (int(within1), int(within2)))
   % pylab.gca().set_yticks([])
   % 


   \begin{frame}
   \frametitle{Rules of thumb}
   \begin{center}
   \resizebox{!}{2.7in}{\includegraphics{./images/inline/7776405d1a.pdf}}    
   \end{center}

   \end{frame}

   %CODE
       % from matplotlib import rc
   % import pylab, numpy as np, sys
   % np.random.seed(0);import random; random.seed(0)
   % sys.path.append('/private/var/folders/dq/4_9bwd013ln6vvf_q110mwrh0000gn/T/tmpi3DZIt')
   % f=pylab.gcf(); f.set_size_inches(8.0,6.0)
   % datadir ='/private/var/folders/dq/4_9bwd013ln6vvf_q110mwrh0000gn/T/tmpi3DZIt/data'
   % import pylab, numpy as np
   % Y = np.random.exponential(size=5000) * 2 + np.random.uniform(size=5000) * 2.
   % pylab.clf()
   % mean = Y.mean()
   % SD = np.sqrt(((Y - Y.mean())**2).mean())
   % pylab.hist(Y, bins=30, alpha=0.5)
   % pylab.arrow(mean, 70, 0, -70,color='r', linewidth=3)
   % pylab.arrow(mean-SD, 70, 0, -70,color='y', linewidth=3)
   % pylab.arrow(mean+SD, 70, 0, -70,color='y', linewidth=3)
   % pylab.arrow(mean-2*SD, 70, 0, -70,color='y', linewidth=3)
   % pylab.arrow(mean+2*SD, 70, 0, -70,color='y', linewidth=3)
   % standY = (Y - mean) / SD
   % within1 = (np.fabs(standY) <= 1).sum() * 1. / Y.shape[0] * 100
   % within2 = (np.fabs(standY) <= 2).sum() * 1. / Y.shape[0] * 100
   % pylab.title('Within 1 SD %d %%, Within 2 SD %d %%' % (int(within1), int(within2)))
   % pylab.gca().set_xlim([-2,15])
   % pylab.gca().set_yticks([])
   % 


   \begin{frame}
   \frametitle{Rules of thumb}
   \begin{center}
   \resizebox{!}{2.7in}{\includegraphics{./images/inline/a8adde404e.pdf}}    
   \end{center}

   \end{frame}

   %%%%%%%%%%%%%%%%%%%%%%%%%%%%%%%%%%%%%%%%%%%%%%%%%%%%%%%%%%%%

   \begin{frame} \frametitle{Root Mean Square}

   \begin{block}
   {Definition}

   $$
   \text{r.m.s. (list)} = \sqrt{\text{average of (entries$^2$)}}
   $$


   \end{block}


   \begin{block}
   {Example}

   $$
   \begin{aligned}
   \text{r.m.s. ([0,5,8,-3])} &= \sqrt{\frac{0^2+5^2+8^2+(-3)^2}{4}} \\
   &= \sqrt{\frac{0+25+64+9}{4}} \\
   &= \sqrt{98} / 2 \\
   & \approx 4.9
   \end{aligned}
   $$


   \end{block}
   \end{frame}

   %%%%%%%%%%%%%%%%%%%%%%%%%%%%%%%%%%%%%%%%%%%%%%%%%%%%%%%%%%%%

   \begin{frame} \frametitle{Root Mean Square}

   \begin{block}
   {$\Sigma$ notation}

   \begin{itemize}
   \item Call our list $X=[X_1, \dots, X_n]$.

   \item Then,
   $$
   \text{r.m.s.}(X) = \sqrt{\frac{1}{n}\sum_{i=1}^n X_i^2}$$
   \end{itemize}



   \end{block}
   \end{frame}

   %%%%%%%%%%%%%%%%%%%%%%%%%%%%%%%%%%%%%%%%%%%%%%%%%%%%%%%%%%%%

   \begin{frame} \frametitle{Standard deviation}

   \begin{block}
   {Computing the SD}

   \begin{itemize}
   \item    Given a list, define
   $$
   \text{deviation from average(list)} = \text{entry - average(list)}.
   $$

   \item    The SD is computed as
   $$
   \text{SD (list)} = \text{r.m.s.(deviation from average(list))}
   $$

   \end{itemize}



   \end{block}

   \begin{block}
   {$\Sigma$ notation}

   \begin{itemize}
   \item Call our list $X=[X_1, \dots, X_n]$.

   \item Then,
   $$
   \text{SD}(X) = \sqrt{\frac{1}{n}\sum_{i=1}^n (X_i-\bar{X})^2}$$
   \end{itemize}



   \end{block}
   \end{frame}

   %%%%%%%%%%%%%%%%%%%%%%%%%%%%%%%%%%%%%%%%%%%%%%%%%%%%%%%%%%%%

   \begin{frame} \frametitle{Standard deviation}

   \begin{block}
   {Example: SD([20,30,25,25])}

   \begin{description}
   \item[Step 1: compute the average]
   $$\text{average} = \frac{20+30+25+25}{4} = \frac{100}{4} = 25.$$
   \item[Step 2: compute deviation from average]
   $$
   \begin{aligned}
   \lefteqn{\text{deviation from average}([20,30,25,25])} \\
   & \qquad = [20-25,30-25,25-25,25-25] \\
   & \qquad = [-5,5,0,0].
   \end{aligned}
   $$
   \end{description}
   \end{block}
   \end{frame}

   %%%%%%%%%%%%%%%%%%%%%%%%%%%%%%%%%%%%%%%%%%%%%%%%%%%%%%%%%%%%

   \begin{frame} \frametitle{Standard deviation}

   \begin{block}
   {Example continued}

   \begin{description}
   \item[Step 3: compute r.m.s of deviation from average]
   $$
   \begin{aligned}
   \text{SD} &= \text{r.m.s}([-5,5,0,0]) \\
   &= \sqrt{\frac{(-5)^2+5^2+0^2+0^2}{4}} \\
   &= \sqrt{\frac{50}{4}} \approx 3.5
   \end{aligned}
   $$
   \end{description}
   \end{block}
   \end{frame}

   %%%%%%%%%%%%%%%%%%%%%%%%%%%%%%%%%%%%%%%%%%%%%%%%%%%%%%%%%%%%

   \begin{frame} \frametitle{Standard deviation}

   \begin{block}
   {SD versus SD$^+$}

   \begin{itemize}
   \item    Some calculators (and software) compute a different
   version of SD than we will use.

   \item The difference depends on the length of the list.

   \item If the length of the list is $n$, then
   $$
   \begin{aligned}
   \lefteqn{\text{SD of a list of $n$ numbers}} \\
   & \qquad = \sqrt{\frac{n-1}{n}} \times \text{SD$^+$ of same list of $n$ numbers}
   \end{aligned}
   $$
   \item Or, in $\Sigma$ notation with list $X=[X_1,\dots, X_n]$
   $$
   \text{SD}^+   = \sqrt{\frac{1}{n-1}\sum_{i=1}^n (X_i-\bar{X})^2}$$
   $$
   \end{itemize}
   \end{block}
   \end{frame}

   %%%%%%%%%%%%%%%%%%%%%%%%%%%%%%%%%%%%%%%%%%%%%%%%%%%%%%%%%%%%

   \begin{frame} \frametitle{Standard deviation}

   \begin{block}
   {Example: SD$^+$([20,30,25,25]) }

   $$
   \begin{aligned}
   \text{SD}^+([20,30,25,25]) &= \sqrt{\frac{4}{3}} \times \text{SD}([20,30,25,25])  \\
   &= \sqrt{\frac{4}{3}} \times \sqrt{\frac{50}{4}} \\
   &= \sqrt{\frac{50}{3}} \\
   &\approx 4.1
   \end{aligned}
   $$
   \end{block}
   \end{frame}

   %%%%%%%%%%%%%%%%%%%%%%%%%%%%%%%%%%%%%%%%%%%%%%%%%%%%%%%%%%%%

   \begin{frame} \frametitle{Changing location / scale}

   \begin{block}
   {Example: temperature conversion}
   \begin{itemize}
   \item    Suppose you are told that the average max temperature in
   Palo Alto
   for March 28 over the last 20 years is 70$^{\circ} F$ with an SD
   of 6 $^{\circ} F$.
   \item What would the average and SD be if you
   used $^{\circ} C$ instead?
   \item The rule for converting Farhenheit to Celsius is
   $$ \text{temp in $^{\circ} C$} = (\text{temp in $^{\circ} F$} - 32^{\circ} F) \times \frac{5^{\circ} C}{9^{\circ} F}
   $$
   \end{itemize}
   \end{block}
   \end{frame}

   %%%%%%%%%%%%%%%%%%%%%%%%%%%%%%%%%%%%%%%%%%%%%%%%%%%%%%%%%%%%

   \begin{frame} \frametitle{Changing location / scale}

   \begin{block}
   {Example: temperature conversion}
   \begin{itemize}
   \item Subtracting $32 ^{\circ} F$ changes the mean of our list by 32, but
   does not change the SD. ({\sc Check this})
   \item Multiplying this new list by $5/9$ changes the SD by a factor of 5/9.
   ({\sc Check this})
   \item Answer: new mean is $(70-32)*5/9\approx 21^{\circ} C$ with an
   SD of $6*5/9 \approx 3.3 ^{\circ} C$.
   \end{itemize}
   \end{block}
   \end{frame}

   %%%%%%%%%%%%%%%%%%%%%%%%%%%%%%%%%%%%%%%%%%%%%%%%%%%%%%%%%%%%

   \begin{frame} 

   \end{frame}

   \end{document}
