   \documentclass[handout]{beamer}



   \mode<presentation>
   {
     \usetheme{PaloAlto}
   \setbeamertemplate{footline}[page number]

     \setbeamercolor{sidebar}{bg=white, fg=black}
     \setbeamercolor{frametitle}{bg=white, fg=black}
     % or ...
     \setbeamercolor{logo}{bg=white}
     \setbeamercolor{block body}{parent=normal text,bg=white}
     \setbeamercolor{author in sidebar}{fg=black}
     \setbeamercolor{title in sidebar}{fg=black}


     \setbeamercolor*{block title}{use=structure,fg=structure.fg,bg=structure.fg!20!bg}
     \setbeamercolor*{block title alerted}{use=alerted text,fg=alerted text.fg,bg=alerted text.fg!20!bg}
     \setbeamercolor*{block title example}{use=example text,fg=example text.fg,bg=example text.fg!20!bg}


     \setbeamercolor{block body}{parent=normal text,use=block title,bg=block title.bg!50!bg}
     \setbeamercolor{block body alerted}{parent=normal text,use=block title alerted,bg=block title alerted.bg!50!bg}
     \setbeamercolor{block body example}{parent=normal text,use=block title example,bg=block title example.bg!50!bg}

     % or ...

     \setbeamercovered{transparent}
     % or whatever (possibly just delete it)
     \logo{\resizebox{!}{1.5cm}{\href{\basename{R}}{\includegraphics{image}}}}
   }

   \mode<handout>
   {
     \usetheme{PaloAlto}
     \usecolortheme{default}
     \setbeamercolor{sidebar}{bg=white, fg=black}
     \setbeamercolor{frametitle}{bg=white, fg=black}
     % or ...
     \setbeamercolor{logo}{bg=white}
     \setbeamercolor{block body}{parent=normal text,bg=white}
     \setbeamercolor{author in sidebar}{fg=black}
     \setbeamercolor{title in sidebar}{fg=black}
     \setbeamercovered{transparent}
     % or whatever (possibly just delete it)
     \logo{}
   }

   \usepackage{epsdice}
   \usepackage[latin1]{inputenc}
   \usepackage{graphics}
   \usepackage{amsmath,eepic,epic}

   \newcommand{\figslide}[3]{
   \begin{frame}
   \frametitle{#1}
     \begin{center}
     \resizebox{!}{2.7in}{\includegraphics{#2}}    
     \end{center}
   {#3}
   \end{frame}
   }

   \newcommand{\fighslide}[4]{
   \begin{frame}
   \frametitle{#1}
     \begin{center}
     \resizebox{!}{#4}{\includegraphics{#2}}    
     \end{center}
   {#3}
   \end{frame}
   }

   \newcommand{\figwref}[1]{
   \href{#1}{\tiny \tt #1}}

   \newcommand{\B}[1]{\beta_{#1}}
   \newcommand{\Bh}[1]{\widehat{\beta}_{#1}}
   \newcommand{\V}{\text{Var}}
   \newcommand{\Cov}{\text{Cov}}
   \newcommand{\Vh}{\widehat{\V}}
   \newcommand{\s}{\sigma}
   \newcommand{\sh}{\widehat{\sigma}}

   \newcommand{\argmax}[1]{\mathop{\text{argmax}}_{#1}}
   \newcommand{\argmin}[1]{\mathop{\text{argmin}}_{#1}}
   \newcommand{\Ee}{\mathbb{E}}
   \newcommand{\Pp}{\mathbb{P}}
   \newcommand{\real}{\mathbb{R}}
   \newcommand{\Ybar}{\overline{Y}}
   \newcommand{\Yh}{\widehat{Y}}
   \newcommand{\Xbar}{\overline{X}}
   \newcommand{\Tr}{\text{Tr}}


   \newcommand{\model}{{\cal M}}

   \newcommand{\figvskip}{-0.7in}
   \newcommand{\fighskip}{-0.3in}
   \newcommand{\figheight}{3.5in}

   \newcommand{\Rcode}[1]{{\bf \tt #1 }}
   \newcommand{\Rtcode}[1]{{\tiny \bf \tt #1 }}
   \newcommand{\Rscode}[1]{{\small \bf \tt #1 }}

   \newcommand{\RR}{{\tt R} \;}
   \newcommand{\basename}[1]{http://stats60.stanford.edu/#1}
   \newcommand{\dataname}[1]{\basename{data/#1}}
   \newcommand{\Rname}[1]{\basename{R/#1}}

   \newcommand{\mycolor}[1]{{\color{blue} #1}}
   \newcommand{\basehref}[2]{\href{\basename{#1}}{\mycolor{#2}}}
   \newcommand{\Rhref}[2]{\href{\basename{R/#1}}{\mycolor{#2}}}
   \newcommand{\datahref}[2]{\href{\dataname{#1}}{\mycolor{#2}}}
   \newcommand{\X}{\pmb{X}}
   \newcommand{\Y}{\pmb{Y}}
   \newcommand{\be}{\pmb{varepsilon}}
   \newcommand{\logit}{\text{logit}}


   \title{Statistics 60: Introduction to Statistical Methods}
   \subtitle{Chapter 3: Histogram / Descriptive Graphics} 
   \author{}% {Jonathan Taylor \\
   %Department of Statistics \\
   %Stanford University
   %}


   \begin{document}

   \begin{frame}
   \titlepage
   \end{frame}

   %%%%%%%%%%%%%%%%%%%%%%%%%%%%%%%%%%%%%%%%%%%%%%%%%%%%%%%%%%%%

   \begin{frame} \frametitle{Descriptive statistics}

   \begin{block}
   {Different types of data}
   \begin{itemize}

   \item Categorical data (e.g. what type of high school).
   \item Ordinal data (e.g. number of missed lectures).
   \item Continuous data (e.g. age of incoming freshmen).
   \item Categorical also called {\em qualitative}
   \item Ordinal also called {\em discrete quantitative}
   \end{itemize}
   \end{block}
   \end{frame}

   %%%%%%%%%%%%%%%%%%%%%%%%%%%%%%%%%%%%%%%%%%%%%%%%%%%%%%%%%%%%

   \begin{frame} \frametitle{Summarizing categorical data}

   \begin{block}
   {Pie graph}
   \begin{itemize}

   \item Each category has a sector of the pie.
   \item Area of wedges proportional to percentage.
   \item If 58 \% went to public school, wedge for public school has angle
   $$
   \frac{58}{100} * 360^{\circ} \approx 210^{\circ}
   $$
   \end{itemize}
   \end{block}
   \end{frame}

   %CODE
       % from matplotlib import rc
   % import pylab, numpy as np, sys
   % np.random.seed(0);import random; random.seed(0)
   % sys.path.append('/private/var/folders/dq/4_9bwd013ln6vvf_q110mwrh0000gn/T/tmpw1Fubh')
   % f=pylab.gcf(); f.set_size_inches(8.0,6.0)
   % datadir ='/private/var/folders/dq/4_9bwd013ln6vvf_q110mwrh0000gn/T/tmpw1Fubh/data'
   % fracs = [10.2, 57.6, 31.8, 0.4]
   % labels = ['International', 'Public', 'Private', 'Home School']
   % pylab.pie(fracs, labels=labels, autopct='%1.0f%%', shadow=True)
   % 


   \begin{frame}
   \frametitle{Where Stanford undergrads went to high school}
   \begin{center}
   \resizebox{!}{2.7in}{\includegraphics{./images/inline/27f72ef082.pdf}}    
   \end{center}

   \end{frame}

   %%%%%%%%%%%%%%%%%%%%%%%%%%%%%%%%%%%%%%%%%%%%%%%%%%%%%%%%%%%%

   \begin{frame} \frametitle{Summarizing ordinal data}

   \begin{block}
   {Missed lectures}
   \begin{tabular}{cc}
   Courses & Percentage \\ \hline
   0 & 75 \% \\
   1 & 7 \% \\
   2 & 3 \% \\
   3 & 4 \% \\
   4 & 6 \% \\
   5 & 5 \% \\
   \end{tabular}
   \end{block}
   \end{frame}

   %CODE
       % from matplotlib import rc
   % import pylab, numpy as np, sys
   % np.random.seed(0);import random; random.seed(0)
   % sys.path.append('/private/var/folders/dq/4_9bwd013ln6vvf_q110mwrh0000gn/T/tmpw1Fubh')
   % f=pylab.gcf(); f.set_size_inches(8.0,6.0)
   % datadir ='/private/var/folders/dq/4_9bwd013ln6vvf_q110mwrh0000gn/T/tmpw1Fubh/data'
   % pylab.bar(range(6),[75,7,3,4,6,5], width=0.1); pylab.show()
   % 


   \begin{frame}
   \frametitle{Number of missed lectures.}
   \begin{center}
   \resizebox{!}{2.7in}{\includegraphics{./images/inline/37a301b080.pdf}}    
   \end{center}

   \end{frame}

   %%%%%%%%%%%%%%%%%%%%%%%%%%%%%%%%%%%%%%%%%%%%%%%%%%%%%%%%%%%%

   \begin{frame} \frametitle{Summarizing continuous data}

   \begin{block}
   {California age demographics}
   \begin{tabular}{l|rr}
   Age group & Count & Percentage \\ \hline
   0-20 & 10000000 & 29\%\\
   20-55 & 17500000 & 17500000 / 34000000 =  52\% \\
   55-75 & 4500000 & 13\%\\
   75+ & 2000000 & 6\%\\ \hline
   Total & 34000000 & 100 \%
   \end{tabular}
   \end{block}
   \end{frame}

   %%%%%%%%%%%%%%%%%%%%%%%%%%%%%%%%%%%%%%%%%%%%%%%%%%%%%%%%%%%%

   \begin{frame} 

   \begin{block}
   {Summarizing a continuous variable}
   \begin{itemize}

   \item Special type of bar graph: histogram.
   \item Areas of bars correspond to percentage.
   \item Total percentage is 100\%.
   \item Height of bars is {\em density}.
   \item Area is {\em density} * {\em width}.
   \end{itemize}
   \end{block}
   \end{frame}

   %CODE
       % from matplotlib import rc
   % import pylab, numpy as np, sys
   % np.random.seed(0);import random; random.seed(0)
   % sys.path.append('/private/var/folders/dq/4_9bwd013ln6vvf_q110mwrh0000gn/T/tmpw1Fubh')
   % f=pylab.gcf(); f.set_size_inches(8.0,6.0)
   % datadir ='/private/var/folders/dq/4_9bwd013ln6vvf_q110mwrh0000gn/T/tmpw1Fubh/data'
   % from denshist import denshist, patchopt
   % import pylab
   % bins = [0,20,55,75,100]
   % count = [29,52,13,6]
   % denshist(count, bins, **patchopt)
   % pylab.gca().set_xlim([0,100])
   % pylab.gca().set_ylim([0,2.])
   % pylab.gca().set_ylabel('Percent per year')
   % pylab.gca().set_xlabel('Age (years)')
   % 


   \begin{frame}
   \frametitle{Histogram}
   \begin{center}
   \resizebox{!}{2.7in}{\includegraphics{./images/inline/2733107db9.pdf}}    
   \end{center}

   \end{frame}

   %%%%%%%%%%%%%%%%%%%%%%%%%%%%%%%%%%%%%%%%%%%%%%%%%%%%%%%%%%%%

   \begin{frame} 

   \begin{block}
   {Histogram example}

   $$
   \text{Percentage in 20-55 age group} = \left(1.5 \, \frac{\%}{\text{year}} \right) * \left( 35 \, \text{years} \right) \approx 52 \%.
   $$

   \end{block}
   \end{frame}

   %%%%%%%%%%%%%%%%%%%%%%%%%%%%%%%%%%%%%%%%%%%%%%%%%%%%%%%%%%%%

   \begin{frame} \frametitle{Cautions about visualization}

   \begin{figure}
   \centering
   \resizebox{!}{2.0in}{\includegraphics{figs/descriptive/incomehist}}
   \end{figure}

   \href{http://en.wikipedia.org/wiki/Household_income_in_the_United_States}{Wikipedia}
   \end{frame}

   %CODE
       % from matplotlib import rc
   % import pylab, numpy as np, sys
   % np.random.seed(0);import random; random.seed(0)
   % sys.path.append('/private/var/folders/dq/4_9bwd013ln6vvf_q110mwrh0000gn/T/tmpw1Fubh')
   % f=pylab.gcf(); f.set_size_inches(8.0,6.0)
   % datadir ='/private/var/folders/dq/4_9bwd013ln6vvf_q110mwrh0000gn/T/tmpw1Fubh/data'
   % import matplotlib.mlab as ML
   % A = ML.csv2rec('%s/householdincome2006.csv' % datadir, delimiter=';')
   % from denshist import denshist, patchopt
   % breaks = np.array([0, 10,20,30,40,50,60,70,80,90,100,150,200,np.inf])*1000
   % count = []
   % for l, u in zip(breaks, breaks[1:]):
   %     g = (A['lower'] < u) * (A['upper'] >= l)
   %     count.append(A['count'][g].sum())
   % breaks[-1] = 300000
   % denshist(count, breaks, **patchopt)
   % #denshist(A['count'], list(A['lower']) + [300000], **patchopt)
   % pylab.gca().set_yticks([])
   % pylab.gca().set_xticks(np.arange(13)*25000)
   % pylab.gca().set_xticklabels(np.arange(13)*25)
   % pylab.gca().set_xlabel('Income (1000s)')
   % 


   \begin{frame}
   \frametitle{Corrected plot}
   \begin{center}
   \resizebox{!}{2.7in}{\includegraphics{./images/inline/d833554974.pdf}}    
   \end{center}
   http://pubdb3.census.gov/macro/032007/hhinc/new06_000.htm
   \end{frame}

   %%%%%%%%%%%%%%%%%%%%%%%%%%%%%%%%%%%%%%%%%%%%%%%%%%%%%%%%%%%%

   \begin{frame} \frametitle{Cautions about visualization}

   \begin{figure}
   \centering
   \resizebox{!}{2.0in}{\includegraphics{figs/descriptive/phillips1}}
   \end{figure}

   \href{http://lilt.ilstu.edu/jpda/}{http://lilt.ilstu.edu/jpda/}
   \end{frame}

   %%%%%%%%%%%%%%%%%%%%%%%%%%%%%%%%%%%%%%%%%%%%%%%%%%%%%%%%%%%%

   \begin{frame} \frametitle{Cautions about visualization}

   \begin{figure}
   \centering
   \resizebox{!}{2.0in}{\includegraphics{figs/descriptive/education}}
   \end{figure}

   \href{http://lilt.ilstu.edu/jpda/}{http://lilt.ilstu.edu/jpda/}
   \end{frame}

   %%%%%%%%%%%%%%%%%%%%%%%%%%%%%%%%%%%%%%%%%%%%%%%%%%%%%%%%%%%%

   \begin{frame} \frametitle{Cautions about visualization}

   \begin{figure}
   \centering
   \resizebox{!}{2.0in}{\includegraphics{figs/descriptive/public_school}}
   \end{figure}

   \href{http://lilt.ilstu.edu/jpda/}{http://lilt.ilstu.edu/jpda/}
   \end{frame}

   %%%%%%%%%%%%%%%%%%%%%%%%%%%%%%%%%%%%%%%%%%%%%%%%%%%%%%%%%%%%

   \begin{frame} \frametitle{Cautions about visualization}

   \begin{figure}
   \centering
   \resizebox{!}{2.0in}{\includegraphics{figs/descriptive/clutter}}
   \end{figure}

   \href{http://lilt.ilstu.edu/jpda/}{http://lilt.ilstu.edu/jpda/}
   \end{frame}

   %%%%%%%%%%%%%%%%%%%%%%%%%%%%%%%%%%%%%%%%%%%%%%%%%%%%%%%%%%%%

   \begin{frame} 

   \end{frame}

   \end{document}
